\documentclass[12pt, titlepage]{article}

\usepackage{amsmath, mathtools}

\usepackage[round]{natbib}
\usepackage{amsfonts}
\usepackage{amssymb}
\usepackage{graphicx}
\usepackage{colortbl}
\usepackage{xr}
\usepackage{hyperref}
\usepackage{longtable}
\usepackage{xfrac}
\usepackage{tabularx}
\usepackage{float}
\usepackage{siunitx}
\usepackage{booktabs}
\usepackage{multirow}
\usepackage[section]{placeins}
\usepackage{caption}
\usepackage{fullpage}
\usepackage{makecell}

\hypersetup{
bookmarks=true,     % show bookmarks bar?
colorlinks=true,       % false: boxed links; true: colored links
linkcolor=red,          % color of internal links (change box color with linkbordercolor)
citecolor=blue,      % color of links to bibliography
filecolor=magenta,  % color of file links
urlcolor=cyan          % color of external links
}

\usepackage{array}

\externaldocument{../../SRS/SRS}


\begin{document}

\title{Module Interface Specification for SCEC (Solar Cooker Energy Calculator)}

\author{Deesha Patel}

\date{\today}

\maketitle

\pagenumbering{roman}

\section{Revision History}

\begin{tabularx}{\textwidth}{p{3cm}p{2cm}X}
\toprule {\bf Date} & {\bf Version} & {\bf Notes}\\
\midrule
March 17, 2023 & 0.1 & Initial Release\\
March 19, 2023 & 0.2 & Updates according to the comments \\ 
March 19, 2023 & 0.3 & Updates according to the issues \\ 
\bottomrule
\end{tabularx}

~\newpage

\section{Symbols, Abbreviations and Acronyms}

See \href{https://github.com/DeeshaPatel/CAS-741-Solar-Cooker/blob/c7cc1be3611cae9110b84940b64ef40c7d29aa02/docs/SRS/SRS.pdf}{SRS} Documentation for symbols, abbreviations and acronyms. \\ 

\renewcommand{\arraystretch}{1.2}
\begin{tabular}{l l} 
  \toprule		
  \textbf{symbol} & \textbf{description}\\
  \midrule 
  c & Condition\\
  en & Energy \\
  energySeq & Energy Sequence\\
  $E_f$ & Energy of fluid \\
  $ewat$ & Calculated Energy watt \\ 
  ODEs & Ordinary Differential Equations \\ 
  param & Parameters\\
  r & Rule  \\
  s & Size of sequence \\
  SCEC & Solar Cooker Energy Calculator \\
  temp & Temperature value \\
  tempSeq & Temperature Sequence \\
  \bottomrule
\end{tabular}\\

\newpage

\tableofcontents

\listoftables

\newpage

\pagenumbering{arabic}

\section{Introduction}

The following document details the Module Interface Specifications for
SCEC (Solar Cooker Energy Calculator). This document specifies how every module is interfacing with every other parts. 

Complementary documents include the \href{https://github.com/DeeshaPatel/CAS-741-Solar-Cooker/blob/2a6c0175891c01960d83cb99b73a762a9b2d2508/docs/SRS/SRS.pdf}{System Requirement Specifications}
and \href{https://github.com/DeeshaPatel/CAS-741-Solar-Cooker/blob/c7cc1be3611cae9110b84940b64ef40c7d29aa02/docs/Design/SoftArchitecture/MG.pdf}{Module Guide}.  The full documentation and implementation can be
found at \href{https://github.com/DeeshaPatel/CAS-741-Solar-Cooker.git}{Github repository for SCEC}.

\section{Notation}


The structure of the MIS for modules comes from \citet{HoffmanAndStrooper1995},
with the addition that template modules have been adapted from
\cite{GhezziEtAl2003}.  The mathematical notation comes from Chapter 3 of
\citet{HoffmanAndStrooper1995}.  For instance, the symbol := is used for a
multiple assignment statement and conditional rules follow the form $(c_1
\Rightarrow r_1 | c_2 \Rightarrow r_2 | ... | c_n \Rightarrow r_n )$.

The following table summarizes the primitive data types used by SCEC. 

\begin{center}
\renewcommand{\arraystretch}{1.2}
\noindent 
\begin{tabular}{l l p{7.5cm}} 
\toprule 
\textbf{Data Type} & \textbf{Notation} & \textbf{Description}\\ 
\midrule
character & char & a single symbol or digit\\
integer & $\mathbb{Z}$ & a number without a fractional component in (-$\infty$, $\infty$) \\
natural number & $\mathbb{N}$ & a number without a fractional component in [1, $\infty$) \\
real & $\mathbb{R}$ & any number in (-$\infty$, $\infty$)\\
\bottomrule
\end{tabular} 
\end{center}

\noindent
The specification of \progname \ uses some derived data types: sequences, strings, and
tuples. Sequences are lists filled with elements of the same data type. Strings
are sequences of characters. Tuples contain a list of values, potentially of
different types. In addition, \progname \ uses functions, which
are defined by the data types of their inputs and outputs. Local functions are
described by giving their type signature followed by their specification.

\section{Module Decomposition}

The following table is taken directly from the Module Guide document for this project.

\begin{table}[h!]
\centering
\begin{tabular}{p{0.3\textwidth} p{0.6\textwidth}}
\toprule
\textbf{Level 1} & \textbf{Level 2}\\
\midrule
{Hardware-Hiding Module} & ~ \\
\midrule

\multirow{7}{0.3\textwidth}{Behaviour-Hiding Module} 
& Constant Value Module\\ 
& Energy Equation Module\\
& Input Format Module\\
& Input Parameter Module\\
& Output Module\\
& Control Module\\
& Temperature ODE Module\\
\midrule

\multirow{3}{0.3\textwidth}{Software Decision Module} 
& ODE Solver Module\\
& Plotting Result Module\\
& Sequence Data Structure Module\\
\bottomrule

\end{tabular}
\caption{Module Hierarchy}
\label{TblMH}
\end{table}


\newpage
~\newpage

\section{MIS of Constant Value Module} \label{Constant_Module} 

\subsection{Module}

ConstValueParams

\subsection{Uses}
None

\subsection{Syntax}

\subsubsection{Exported Constants}
\begin{center}
\begin{tabular}{p{2.5cm} p{1cm} p{3cm} p{7cm}}
stefan\_const & \si{\sigma} & 5.670374419e-08 & Stefan-Boltzman constant \\
h\_t\_int3 & $h_\text{t\_int3}$ & 4.0 & Heat flux from Lid to Inner area of container\\
h\_ref\_int2 & $h_\text{ref\_int2}$ & 4.4 & Heat flux from reflector to Inner area of box \\ 
h\_ref\_f & $h_\text{ref\_f}$ & 4.0 & Heat flux from reflector to fluid \\
c\_ref & $c_\text{ref}$ & 900 & Specific heat capacity of reflector \\
c\_f & $c_\text{f}$ & 4190 & Specific heat capacity of fluid \\
m\_ref & $m_\text{ref}$ & 0.2 & Mass of reflector \\
m\_f & $m_\text{f}$ & 2.0 & Mass of fluid \\
t\_g & $\tau_\text{g}$ & 0.48 & Transmittivity of glass \\ 
p & $\rho$ & 0.89 & Reflectivity of glass \\ 
\end{tabular}
\end{center}

\subsubsection{Exported Access Programs}

None

\subsection{Semantics}

\subsubsection{State Variables}

None


\subsubsection{Environment Variables}

None

\subsubsection{Assumptions}

None

\subsubsection{Access Routine Semantics}

None

\subsubsection{Local Functions}

None

\subsubsection{Considerations}
Note: These constants are as per the SRS document, so constant parameters and values may change according to the implementation if required.  

\newpage

\section{MIS of Energy Equation Module} \label{Energy_Equation_Module} 

\subsection{Module}

energy\_calculation

\subsection{Uses}

\begin{itemize}
    \item Input Parameter Module
    \item Constant Value Module
\end{itemize}

\subsection{Syntax}

\subsubsection{Exported Constants}
None

\subsubsection{Exported Access Programs}

\begin{center}
\begin{tabular}{p{1.9cm} p{5cm} p{4cm} p{3.5cm}}
\hline
\textbf{Name} & \textbf{In} & \textbf{Out} & \textbf{Exceptions} \\
\hline
energyWat & sequence of $\mathbb{R}$ & sequence of $\mathbb{R}$ & ValueError \\   
\hline
\end{tabular}
\end{center}

\subsection{Semantics}

\subsubsection{State Variables}

None

\subsubsection{Environment Variables}

None  

\subsubsection{Assumptions}

The Energy Equation Module is called through the Control Module, ensuring that Temperature ODE Module has been called before Energy Equation Module and fluid temperature values are calculated to give input to the Energy Equation Module.

\subsubsection{Access Routine Semantics}
This module satisfies R5 from the SRS. \\ \\
\noindent \textbf{energyWat(temp):}
\begin{itemize}
\item transition: None
\item output: The energy equation module returns the sequence of energy: \\ out := $ < i \in \mathbb{N}, 0 \le i < |s-1|$ $ | $ $m_f c_f (temp[i] - T_\text{init}) > $  , where $m_f$ and $c_f$ are constant value for mass of fluid and capacity of fluid respectively, $s$ = size of sequence

\item exception: exc := \\ \\
 \begin{tabular}{p{5cm} p{3.5cm} p{5.5cm}}
 \hline
 \textbf{Expression} & \textbf{Exception} & \textbf{Description} \\
  \hline
  $\neg ( \triangle T > 0 ) $ & ValueError & Valid temperature value should positive only. \\ 
  \hline
    ($E_f = 0 \vee E_f \notin \mathbb{R}$) & ValueError & Energy of fluid should be real and non zero number.   \\
    \hline 
        ($ewat = \emptyset $) & ValueError & Energy sequence should have at least one value, not an empty sequence. \\ 
    \hline
    ($temp = \emptyset $) & ValueError & Temperature sequence should not be empty. \\ 
    \hline
 \end{tabular}
 
\end{itemize}

\subsubsection{Local Functions}

None

\newpage

\section{MIS of Input Format Module} \label{Input_Format_Module} 

\subsection{Module}

format\_input 

\subsection{Uses}

\begin{itemize}
    \item Input Parameter Module
    \item Hardware Hiding Module
\end{itemize}

\subsection{Syntax}

\subsubsection{Exported Constants}
None 
\subsubsection{Exported Access Programs}

\begin{center}
\begin{tabular}{p{2.5cm} p{3.5cm} p{4cm} p{4cm}}
\hline
\textbf{Name} & \textbf{In} & \textbf{Out} & \textbf{Exceptions} \\
\hline
load\_params & String & sequence of $\mathbb{R}$ & InputError, FileFormatError \\
\hline
\end{tabular}
\end{center}

\subsection{Semantics}

\subsubsection{State Variables}

None

\subsubsection{Environment Variables}

\begin{enumerate}
    \item file: A file containing sequence of strings that provides data related to temperature, Area and other properties.
        \item Windows keyboard: Input Format Module takes the input from the keyboard in the file. 
\end{enumerate}


\subsubsection{Assumptions}

\begin{itemize}
    \item The Control Module call this module for formating input parameters. 
    \item The paramFile contains input starts with '\#' in new line. The order of the inputs should be as below: \\ 
    Line 1: Area of lid \\
    Line 2: Temperature of lid \\
    Line 3: Temperature of fluid \\
    Line 4: Emissivity of lid \\
    Line 5: Area of reflector \\ 
    Line 6: Temperature of reflector \\ 
    Line 7: Emissivity of reflector \\ 
    Line 8: Temperature of glass \\
    Line 9: Area of mass 
\end{itemize}

\subsubsection{Access Routine Semantics}
This module is a function to load, verify and store input data. (R1 and R2 from SRS). \\ \\
\noindent \textbf{load\_params(paramFile):}
\begin{itemize}
\item transition:  paramFile is the file for fetching input values from the file. The following procedure is performed: 
\begin{enumerate}
    \item Verify the format of the file to be .txt. 
    \item Extract the input one by one. 
    \item Verify all inputs, verifyInput(param)
    \item Store inputs to the data structure 
\end{enumerate}
\item output: Give sequence of inputs contains all inputted data under appropriate field names. \\ 
out := params 
\item exception: Data input which does not comply with the data constraints specified in SRS for this project will yield one of the potential exceptions or warning as listed in the appendix of this document. 
\end{itemize}

\subsubsection{Local Functions}

verifyInputs : string, $\mathbb{R}$ \\ 
\textbf{verifyInputs(file, param)}: A function to verify the inputs for SCEC.   
\begin{itemize} 
    \item output: out:= none
    \item exception: exc:= \\ \\ 
     \begin{tabular}{p{5cm} p{3.5cm} p{5.5cm}}
 \hline
 \textbf{Expression} & \textbf{Exception} & \textbf{Description} \\
  \hline
    If file is not in correct format & FileFormatError & Valid input file should be *.txt file only.  \\
    $\neg $($|s|$ = 9) & InputError & Valid size of input sequence should be 9. \\ 
    \hline
 \end{tabular}
 
    See appendix (Table \ref{verify_input}) for all constraints and error message.  
    
\end{itemize}


\newpage
\section{MIS of Input Parameter Module} \label{Input_Parameter_Module} 

\subsection{Module}

parameters

\subsection{Uses}

\begin{itemize}
    \item Hardware Hiding Module
\end{itemize}

\subsection{Syntax}

\subsubsection{Exported Constants}
None 

\subsubsection{Exported Access Programs}

\begin{center}
\begin{tabular}{p{2cm} p{4cm} p{4cm} p{2cm}}
\hline
\textbf{Name} & \textbf{In} & \textbf{Out} & \textbf{Exceptions} \\
\hline
Parameters  & - & - & - \\
\hline
\end{tabular}
\end{center}

\subsection{Semantics}
Parameters is a data structure designed to store the input information entered by the Input Format Module. 
\subsubsection{State Variables}

param := sequence of ( \\
$A_t: \mathbb{R}$, Area of lid \\
$T_t: \mathbb{R}$, Temperature of lid \\ 
$T_f: \mathbb{R}$, Temperature of fluid \\
$e_t: \mathbb{R}$, Emissivity of lid \\
$A_\text{ref}: \mathbb{R}$, Area of reflector \\
$T_\text{ref}: \mathbb{R}$, Temperature of reflector  \\
$e_\text{ref}: \mathbb{R}$, Emissivity of reflector  \\
$T_g: \mathbb{R}$, Temperature of glass   \\ 
$A_m: \mathbb{R}$, Area of mass 
\\ )

\subsubsection{Environment Variables}

None

\subsubsection{Assumptions}

None

\subsubsection{Access Routine Semantics}

\noindent Parameters:
\begin{itemize}
\item transition: This module is a simple data structure for storing the input values formatted by Input Format Module. 
\item output: None
\item exception: exc:= None
\end{itemize}


\subsubsection{Local Functions}

None

\newpage
\section{MIS of Output Module} \label{Output_Format_Module}

\subsection{Module}
output

\subsection{Uses}
\begin{itemize}
    \item Input Parameter Module 
    \item Hardware Hiding Module    
    \item Plotting Result Module
\end{itemize}

\subsection{Syntax}

\subsubsection{Exported Constants}
None 
\subsubsection{Exported Access Programs}


\begin{tabular}{p{2cm} p{5.5cm} p{3cm} p{3.5cm}}
\hline
\textbf{Name} & \textbf{In} & \textbf{Out} & \textbf{Exceptions} \\
\hline
output & \makecell[l]{String, \\ sequence of $\mathbb{R}$, \\ sequence of $\mathbb{R}$, \\ sequence of $\mathbb{N}$ } & Output File & \makecell[l]{MissingValueError, \\ FileAlreadyExistError} \\
\hline
\end{tabular}

\subsection{Semantics}

\subsubsection{State Variables}

None

\subsubsection{Environment Variables}

\begin{enumerate}
    \item file: The file in which the output is saved. 
    \item Window screen: Output Module prints the result in the graph, which is shown to the screen. 
\end{enumerate}


\subsubsection{Assumptions}

The Control Module properly verified values against the constraint.   

\subsubsection{Access Routine Semantics}

\noindent \textbf{output(fileName, tempSeq, energySeq, t):}
\begin{itemize}
\item transition: This module is able to print the output in the file which contains output of the temperature and energy sequence. This output file will be store in the machine and in the form of .out format.  
\item output: out := file
\item exception: exc:= None 
\end{itemize}

\subsubsection{Local Functions}

None

\newpage
\section{MIS of Control Module} \label{SCEC_Control_Module} 

\subsection{Module}

main 

\subsection{Uses}

\begin{itemize}
    \item Constant Value Module
    \item Energy Equation Module
    \item Hardware Hiding Module
    \item Input Format Module 
    \item Output Module 
    \item Temperature ODE Module
    \item ODE Solver Module
    \item Sequence Data Structure Module
\end{itemize}

\subsection{Syntax}

\subsubsection{Exported Constants}
None
\subsubsection{Exported Access Programs}

\begin{center}
\begin{tabular}{p{2cm} p{4cm} p{4cm} p{2cm}}
\hline
\textbf{Name} & \textbf{In} & \textbf{Out} & \textbf{Exceptions} \\
\hline
main & - & Modifies output file & Various \\
\hline
\end{tabular}
\end{center}

\subsection{Semantics}

\subsubsection{State Variables}
\begin{itemize}
    \item init\_temp : sequence of [T1: $\mathbb{R}$, T2: $ \mathbb{R}$]            \# initial temperature values.  
    \item t : vector        \# vector of time. 
    \item temp : sequence of [T1: $\mathbb{R}$, T2: $\mathbb{R}$]   \# sequence 2D for temperatures. 
    \item e\_f : sequence of [E1: $ \mathbb{R}$] \# sequence 1D for energy of fluid. 
\end{itemize}

\subsubsection{Environment Variables}

None

\subsubsection{Assumptions}

None

\subsubsection{Access Routine Semantics}

\noindent main():
\begin{itemize}
\item transition: Control the order of execution of different modules as follow: \\
\begin{itemize}
    \item Set constant value using Constant Value Module (M2, Section \ref{Constant_Module}). 
    \item Set inputted values to the appropriate variables using Input Format Module (M4, Section \ref{Input_Format_Module}). 
    \item Set the time vector using Sequence Data Structure Module (M11, Section \ref{Time_Vector_Module}).
    \item Temperature values for reflector and fluid is calculated using initial conditions by Temperature ODE Module (M8, Section \ref{Temperature_ODEs_Module}). 
    \item Using the previous step output, energy of fluid is calculated in Energy Equation Module (M3, Section \ref{Energy_Equation_Module}).
    \item Output is transferred to the output file with the help of Output Module and internally it also called Plotting Result Module for plotting result on graphs (M6, Section \ref{Output_Format_Module} and M10, Section \ref{Plotting_Result_Module}). 
     
\end{itemize}
\item output: out := None 
\item exception: exc:= Potential exceptions occurs are from different sub-modules only.   
\end{itemize}


\subsubsection{Local Functions}

None


\newpage
\section{MIS of Temperature ODE Module} \label{Temperature_ODEs_Module} 

\subsection{Module}

calculation

\subsection{Uses}

\begin{itemize}
    \item Constant Value Module 
    \item Input Parameter Module
\end{itemize}

\subsection{Syntax}

\subsubsection{Exported Constants}
None

\subsubsection{Exported Access Programs}

\begin{tabular}{p{3cm} p{5cm} p{4cm} p{2.5cm}}
\hline
\textbf{Name} & \textbf{In} & \textbf{Out} & \textbf{Exceptions} \\
\hline
calculateOde & sequence of [$T1: \mathbb{R}$, $T2: \mathbb{R}$], sequence of $\mathbb{R}$ & sequence of [$T1: \mathbb{R}$, $T2: \mathbb{R}$] & TypeError, NameError, MissingValueError, EmptyArrayError, ValueError \\
\hline
\end{tabular}

\subsection{Semantics}

\subsubsection{State Variables}
None
 

\subsubsection{Environment Variables}

None

\subsubsection{Assumptions}

None

\subsubsection{Access Routine Semantics}

\noindent \textbf{calculation(initial\_condition, params):}
\begin{itemize}
\item transition: Temperature is calculated as follows: 
    \begin{itemize}
        \item Calculate and set the value of $in\_t$ which is an inner temperature of the box calculate by performing mean of glass, lid of recipient and reflector temperature. 
        \begin{center}
            $in\_t = \frac{T_\text{glass} + T_\text{lid} + T_\text{ref}}{3}$
        \end{center}
        \item Calculate and store the values of $qs$ using the input $params$. 
    
            $q11 = A_t h_\text{t-int3}(T_\text{t} - T_f)$ \\ 
            $q12 = A_t \sigma \epsilon_t (T_\text{t}^4 - T_f^4)$ \\ 
            $q13 = A_\text{ref} h_\text{ref-int2}(T_\text{int2} - T_\text{ref})$ \\ 
            $q14 = \sum_{i=1}^n \rho A_\text{ref,n} G \tau_g^2 cos (90 - \theta_\text{ref,n})$ \\
            $q15 = A_\text{ref} \sigma \epsilon_\text{ref} (T^4_\text{ref} - T^4_\text{g2})$ \\
            $q16 = A_\text{ref} \sigma \epsilon_\text{ref} (T^4_\text{ref} - T^4_f)$ \\
            $q17 = A_m h_\text{\text{ref}-f}(T_\text{ref} - T_f)$ \\
            
        \item Find the value of $dr$ and $df$ using $qs$ and constant values. \\ \\ 
        $dr = \frac{q13 + 4 q14 - q15 - q16 - q17}{m_r  c_r}$ \\  \\
        $df = \frac{q11 + q12 + q16 + q17}{m_f  c_f}$
        \item Return calculated $dr$ and $df$ as a sequence. 
    \end{itemize}
\item output: 
    Temperature ODE Module give an output of 2D sequence which stores temperature of reflector and fluid. \\  
    out := $s$ \# temperature of reflector and fluid respectively  

    
\item exception: exc:= \\ \\ 
 \begin{tabular}{p{7cm} p{3.5cm} p{4cm}}
 \hline
 \textbf{Expression} & \textbf{Exception} & \textbf{Description} \\
 \hline
     ($\forall i \in [0..|s|-1]$)($\text{initial\_condition}[i] \notin \mathbb{R}$) & TypeError & Valid initial input for the temperature sequence are real numbers. \\
  \hline
      If any of the input is missing  & MissingValueError & Module requires 3 input values: fileName, temperatureSeq, and energySeq \\
    \hline
    If tries to use variable that is not declared. & NameError & Variables those are declared in the module can accessible. \\
    \hline
 \end{tabular}
\end{itemize}

\subsubsection{Local Functions}

verifyTemp : $\mathbb{R}$ \\ 
\textbf{verifyTemp(temp)}:  A function to verify the temperature sequence. 
\begin{itemize}
    \item  output: None 
    \item  exception: \\

 \begin{tabular}{p{4.3cm} p{3.3cm} p{6cm}}
 \hline
 \textbf{Expression} & \textbf{Exception} & \textbf{Description} \\
  \hline
    ($dr < 0 \vee df < 0 $) & ValueError & Valid temperature value should not negative \\
    \hline 
    ($dr = \emptyset \vee df = \emptyset $) & EmptyArrayError &  Temperature sequence should not null \\
    \hline
 \end{tabular}
\end{itemize}

\newpage
\section{MIS of ODE Solver Module} \label{ODE_Solver_Module} 

\subsection{Module}

solver

\subsection{Uses}

\begin{itemize}
    \item Temperature ODE Module
    \item Sequence Data Structure Module
\end{itemize}

\subsection{Syntax}

\subsubsection{Exported Constants}
None 

\subsubsection{Exported Access Programs}

\begin{tabular}{p{1.5cm} p{5.5cm} p{4.5cm} p{3cm}}
\hline
\textbf{Name} & \textbf{In} & \textbf{Out} & \textbf{Exceptions} \\
\hline
solveOde & String, sequence of $\mathbb{R}$, vector, sequence of $\mathbb{R}$ & sequence of [$T1: \mathbb{R}, T2: \mathbb{R}$] & ValueError, TypeError, OverflowError, RuntimeError \\
\hline
\end{tabular}

\subsection{Semantics}

\subsubsection{State Variables}

None

\subsubsection{Environment Variables}

None

\subsubsection{Assumptions}

All input parameters to the $solveOde()$ are correct and verified by the Control Module. 

\subsubsection{Access Routine Semantics}

\noindent solveOde($funcName, init\_cond, t, args$):
\begin{itemize}
\item transition: ODE is calculated as follows: 
\begin{itemize}
    \item Takes specified inputs as a parameter. 
    \item With specified function name (first argument) in solveOde, initial conditions, time interval and extra parameters the solution is to be computed. 
    \item Output is store in the local variable. 
\end{itemize}


\item output: ODE Solver Module give an output of 2D sequence from Temperature ODE Module using programming library. 
\begin{itemize}
    \item out := sequence $s$ 
\end{itemize}

\item exception: exc:= \\ \\ 
\begin{tabular}{p{7cm} p{3.5cm} p{4cm}}
 \hline
 \textbf{Expression} & \textbf{Exception} & \textbf{Description} \\
 \hline
     ($\forall i \in [0..|s|-1]$)($\text{initial\_condition}[i] \notin \mathbb{R} \vee \text{initial\_condition}[i] \notin \mathbb{N} \vee \text{initial\_condition}[i] \notin \mathbb{Z}$) & ValueError & Valid initial input for the temperature sequence are real or natural numbers. \\
  \hline
      solveOde(init\_cond, funcName, t, args)  & TypeError & Module requires 4 input values in order of funcName, init\_cond, t and args.  \\
    \hline
    If solution of solveOde results larger value of temperature than range of double. & OverflowError & Limit of the temperature should be correct. \\
    \hline 
    If the specified function has some problems &  RuntimeError & Function should work properly in order to solve the integration of ODE. \\ 
    \hline
 \end{tabular}
\end{itemize}

\subsubsection{Local Functions}

None
\newpage
\section{MIS of Plotting Result Module} \label{Plotting_Result_Module} 

\subsection{Module}

plot

\subsection{Uses}

\begin{itemize}
    \item Hardware Hiding Module
\end{itemize}

\subsection{Syntax}

\subsubsection{Exported Constants}

None

\subsubsection{Exported Access Programs}

\begin{center}
\begin{tabular}{p{2cm} p{4cm} p{4cm} p{2cm}}
\hline
\textbf{Name} & \textbf{In} & \textbf{Out} & \textbf{Exceptions} \\
\hline
plot & vector, sequence of $\mathbb{R}$ & graph & -  \\
\hline
\end{tabular}
\end{center}

\subsection{Semantics}

\subsubsection{State Variables}

None

\subsubsection{Environment Variables}

Windows screen: As this module display a graph on screen, it uses screen for it. 

\subsubsection{Assumptions}

None

\subsubsection{Access Routine Semantics}

\noindent plot(t, s):
\begin{itemize}
\item transition: Plotting Result Module display the graph using the received input parameters.
\item output: None
\item exception: exc:= 
None
\end{itemize}


\subsubsection{Local Functions}

None

\subsection{Considerations}

This module usually handle by the programming language. For SCEC system, we are using \href{https://matplotlib.org/stable/tutorials/introductory/pyplot.html}{matplotlib} to plot the result. So, exceptions are handled by the language itself. 

\newpage
\section{MIS of Sequence Data Structure Module} \label{Sequence_Data_Structure_Module} 
 

\subsection{Module}

sequential


\subsection{Considerations}

The Sequence Data Structure Module is handled by the programming language. For the purpose of sequences, SCEC is using \href{https://numpy.org}{NumPy}.   

\newpage
\section{Appendix} \label{Appendix}

\begin{table}[!h]
  \caption{Possible errors for input} \label{verify_input}
  \renewcommand{\arraystretch}{1.2}
\noindent \begin{longtable*}{l l l l c} 
  \toprule
  \textbf{Var} & \textbf{Physical Constraints} & \textbf{Error Message} \\
  \midrule 

 $A_\text{ref}$ & $0 < A_\text{ref} \le 1$ & InvalidAreaOfReferenceInputError \\
  $A_m$ & $0 < A_m \le 1 $ & InvalidAreaOfMassInputError \\
  $A_t$ & $0 < A_t \le 1 $ & InvalidAreaOfLidInputError \\
  $T_\text{f}$ & $20 < T_\text{f} < 100$ & InvalidTemperatureOfFluidInputError\\
  $T_\text{ref}$ & $20 < T_\text{ref} < 100$ &InvalidTemperatureOfReferenceInputError \\
  $T_\text{g2}$ & $20 < T_\text{g2} < 100$ &InvalidTemperatureOfGlassInputError \\
  $T_\text{t}$ & $20 < T_\text{t} < 100$ & InvalidTemperatureOfLidInputError\\
  $\epsilon_\text{ref}$ & $0 < \epsilon_\text{ref} < 1$ & InvalidEmittanceOfReferenceInputError\\
   $\epsilon_\text{t}$ & $0 < \epsilon_\text{t} < 1$ & InvalidEmittanceOfLidInputError \\
  
  \bottomrule
\end{longtable*}
\end{table}

\newpage
\bibliographystyle {plainnat}
\bibliography {srs}

\end{document}
