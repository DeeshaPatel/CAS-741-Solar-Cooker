
\documentclass[12pt]{article}

\usepackage{amsmath, mathtools}
\usepackage{amsfonts}
\usepackage{amssymb}
\usepackage{graphicx}
\usepackage{colortbl}
\usepackage{xr}
\usepackage{hyperref}
\usepackage{longtable}
\usepackage{xfrac}
\usepackage{tabularx}
\usepackage{float}
\usepackage{siunitx}
\usepackage{booktabs}
\usepackage{caption}
\usepackage{pdflscape}
\usepackage{afterpage}

\usepackage[round]{natbib}

%\usepackage{refcheck}

\hypersetup{
    bookmarks=true,         % show bookmarks bar?
      colorlinks=true,       % false: boxed links; true: colored links
    linkcolor=red,          % color of internal links (change box color with linkbordercolor)
    citecolor=green,        % color of links to bibliography
    filecolor=magenta,      % color of file links
    urlcolor=cyan           % color of external links
}

% For easy change of table widths
\newcommand{\colZwidth}{1.0\textwidth}
\newcommand{\colAwidth}{0.13\textwidth}
\newcommand{\colBwidth}{0.82\textwidth}
\newcommand{\colCwidth}{0.1\textwidth}
\newcommand{\colDwidth}{0.05\textwidth}
\newcommand{\colEwidth}{0.8\textwidth}
\newcommand{\colFwidth}{0.17\textwidth}
\newcommand{\colGwidth}{0.5\textwidth}
\newcommand{\colHwidth}{0.28\textwidth}

% Used so that cross-references have a meaningful prefix
\newcounter{defnum} %Definition Number
\newcommand{\dthedefnum}{GD\thedefnum}
\newcommand{\dref}[1]{GD\ref{#1}}
\newcounter{datadefnum} %Datadefinition Number
\newcommand{\ddthedatadefnum}{DD\thedatadefnum}
\newcommand{\ddref}[1]{DD\ref{#1}}
\newcounter{theorynum} %Theory Number
\newcommand{\tthetheorynum}{T\thetheorynum}
\newcommand{\tref}[1]{T\ref{#1}}
\newcounter{tablenum} %Table Number
\newcommand{\tbthetablenum}{T\thetablenum}
\newcommand{\tbref}[1]{TB\ref{#1}}
\newcounter{assumpnum} %Assumption Number
\newcommand{\atheassumpnum}{P\theassumpnum}
\newcommand{\aref}[1]{A\ref{#1}}
\newcounter{goalnum} %Goal Number
\newcommand{\gthegoalnum}{P\thegoalnum}
\newcommand{\gsref}[1]{GS\ref{#1}}
\newcounter{instnum} %Instance Number
\newcommand{\itheinstnum}{IM\theinstnum}
\newcommand{\iref}[1]{IM\ref{#1}}
\newcounter{reqnum} %Requirement Number
\newcommand{\rthereqnum}{P\thereqnum}
\newcommand{\rref}[1]{R\ref{#1}}
\newcounter{nfrnum} %NFR Number
\newcommand{\rthenfrnum}{NFR\thenfrnum}
\newcommand{\nfrref}[1]{NFR\ref{#1}}
\newcounter{lcnum} %Likely change number
\newcommand{\lthelcnum}{LC\thelcnum}
\newcommand{\lcref}[1]{LC\ref{#1}}

\usepackage{fullpage}

\newcommand{\deftheory}[9][Not Applicable]
{
\newpage
\noindent \rule{\textwidth}{0.5mm}

\paragraph{RefName: } \textbf{#2} \phantomsection 
\label{#2}

\paragraph{Label:} #3

\noindent \rule{\textwidth}{0.5mm}

\paragraph{Equation:}

#4

\paragraph{Description:}

#5

\paragraph{Notes:}

#6

\paragraph{Source:}

#7

\paragraph{Ref.\ By:}

#8

\paragraph{Preconditions for \hyperref[#2]{#2}:}
\label{#2_precond}

#9

\paragraph{Derivation for \hyperref[#2]{#2}:}
\label{#2_deriv}

#1

\noindent \rule{\textwidth}{0.5mm}

}

\begin{document}


\title{Software Requirements Specification for Solar Cooker Energy Calculation: A Software for calculating heat loss and observed heat} 
\author{Deesha Patel}
\date{\today}
	
\maketitle

~\newpage

\pagenumbering{roman}

\tableofcontents

~\newpage

\section*{Revision History}

\begin{tabularx}{\textwidth}{p{3cm}p{2cm}X}
\toprule {\bf Date} & {\bf Version} & {\bf Notes}\\
\midrule
02/04/2023 & 1.0 & Initial Release\\

\bottomrule
\end{tabularx}

~\newpage

\section{Reference Material}

This section records information for easy reference.

\subsection{Table of Units}

Throughout this document SI (Syst\`{e}me International d'Unit\'{e}s) is employed
as the unit system.  In addition to the basic units, several derived units are
used as described below.  For each unit, the symbol is given followed by a
description of the unit and the SI name.
~\newline

\renewcommand{\arraystretch}{1.2}
%\begin{table}[ht]
  \noindent \begin{tabular}{l l l} 
    \toprule		
    \textbf{symbol} & \textbf{unit} & \textbf{SI}\\
    \midrule 
    \si{\metre} & length & metre\\
    \si{\kilogram} & mass	& kilogram\\
    \si{\second} & time & second\\
    \si{\celsius} & temperature & centigrade\\
    \si{\joule} & energy & joule\\
    \si{\watt} & power & watt (W = \si{\joule\per\second})\\
    \bottomrule
  \end{tabular}
  %	\caption{Provide a caption}
%\end{table}

\subsection{Table of Symbols}

The table that follows summarizes the symbols used in this document along with
their units.  The choice of symbols was made to be consistent with the heat
transfer literature and with existing documentation for solar water heating
systems.  The symbols are listed in alphabetical order.

\renewcommand{\arraystretch}{1.2}
%\noindent \begin{tabularx}{1.0\textwidth}{l l X}
\noindent \begin{longtable*}{l l p{12cm}} \toprule
\textbf{symbol} & \textbf{unit} & \textbf{description}\\
\midrule 

$A_\text{g}$ & \si[per-mode=symbol] {\square\metre} & Area of glass
\\ 

$A_\text{m}$ & \si[per-mode=symbol] {\square\metre} & wet surface area over which heat is transferred in
\\ 

$A_r$ & \si[per-mode=symbol] {\square\metre} & Area of Reflector
\\



$A_\text{ref}$ & $\si[per-mode=symbol] {\square\metre} $ & Reflector area with respect to different reflectors \\

$C_W$ & $\si{\joule\per(\kilogram \celsius)}$ & Specific Heat Capacity of Water \\

$E_W$ & $\si{\joule}$ & Change in heat energy in the Water \\

$f$ & - & Fluid \\

$G$ & \si[per-mode=symbol]{\watt\per\square\metre} & Incidental solar radiation \\

$h$ & \si[per-mode=symbol]{\watt\per\square\metre K} & heat transfer convection coefficient \\

$int2$ & - & Inner 2  \\

$L$ & \si{\metre} & Length \\

$m_W$ & \si{\kilogram} & Mass of Water \\

$n$ & - & Number of Reflector \\

$r$ & - & Reflector \\

$T$ & \si{\celsius} & Temperature\\

$T_W$ & \si{\celsius} & Temperature of Water\\

$\epsilon$ & - & Emittance \\

$\tau$ & - & Transmittivity \\

$\theta$ & - & Reflector Angle \\ 

$\sigma$ & - & Steffan-Boltzman constant \\

$\rho$ & - & Reflectivity \\

\bottomrule
\end{longtable*}

\subsection{Abbreviations and Acronyms}

\renewcommand{\arraystretch}{1.2}
\begin{tabular}{l l} 
  \toprule		
  \textbf{symbol} & \textbf{description}\\
  \midrule 
  A & Assumption\\
  DD & Data Definition\\
  GD & General Definition\\
  GS & Goal Statement\\
  IM & Instance Model\\
  LC & Likely Change\\
  PS & Physical System Description\\
  R & Requirement\\
  SRS & Software Requirements Specification\\
  T & Theoretical Model\\
  \bottomrule
\end{tabular}\\

\newpage

\pagenumbering{arabic}

\section{Introduction}

As fossil fuels adversely affect the environment, many countries like India, where they have good Solar rays throughout the year, have started to implement devices for utilizing solar energy and transforming it into valuable energy. It includes Solar Water heaters, Solar panels, and Solar Cookers. It is indeed a fact that the demand for renewable energy has increased to deal with the problem. Focusing on solar cooker, several design such as Solar Panel Cooker, Solar Parabolic Cooker, and Solar Box Cooker has been proposed to utilize more solar energy. Using internal reflectors, we would like to improve the utilization of solar energy during cooking a food.   

The following section provides an overview of Software Requirement Specification (SRS) for a box-type Solar Cooker. This section explains the purpose of the document, the scope of requirements, the characteristic of the intended reader, and the organization of a document.   


\subsection{Purpose of Document}

We are going to use the Solar Box Cooker in this software. The main purpose of this document is to describe a mathematical model of a Solar Cooker which can provide a calculation for internal reflector in solar box that can help to improve the temperature in the box. It includes a variety of parameters that attempts to define the intended functionality required. Thus, this document provides detailed requirements of the software which will be used in planing for design stage. Therefore, this document is intended to be used as a reference to provide ad hoc access to all information necessary to understand and verify the model. This document describes goals, assumptions, theoretical models, and important definitions to understand the problem. The SRS is abstract because the content here says \emph{what} the problem is. But it does not say anything related to \emph{how} to solve it.

This document will be used as a starting point for subsequent development 
phases, including writing the design specification and the software 
verification and validation plan. The design document will show how the 
requirements are to be realized, including decisions on the numerical 
algorithms and programming environment. The verification and validation plan 
will show the steps that will be used to increase confidence in the software 
documentation and the implementation. Although the SRS fits in a series of 
documents that follow the so-called waterfall model, the actual development 
process is not constrained in any way. Even when the waterfall model is not 
followed, as Parnas and Clements~\cite{ParnasAndClements} point out, the most logical way to 
present the documentation is still to “fake” a rational design process.

\subsection{Scope of Requirements} 

There are Numerous algorithm has been proposed to improve the efficiency of Solar Cooker. The scope of the requirement includes a mathematical model to determine the thermal function of a box-type solar cooker with an internal reflectors. With the help of different inputs, this system calculates the achieved temperature by implementing the proposed solution. However, this project not focusing on more than one iteration for the reflections. 

\subsection{Characteristics of Intended Reader} \label{sec_IntendedReader}

Firstly, Intended Reader or Reviewer should have knowledge of heat transfer theory and radiant solar energy. A person should have completed a Heat Transfer course during their bachelor of engineering (Mechanical Engineering expected). A reader should have knowledge about the coupled differential equation; offered in the Calculus course.        

\subsection{Organization of Document}

The organization of this document follows the template for an SRS for scientific 
computing software proposed by~\cite{Koothoor2013} and \cite{SmithAndLai2005}.
The presentation follows the standard pattern of presenting goals, theories, definitions, 
and assumptions. For readers that would like a more bottom up approach, they can start 
reading the instance models in Section~\ref{sec_instance} and trace back to find any 
additional information they require. The goal statements are refined to the theoretical models, 
and the theoretical models to the instance models. The instance model 
(Section~\ref{sec_instance}) to be solved is referred to as \iref{ewat}. The instance model provides 
the Ordinary Differential Equation (ODE) that model the solar water heating system. 
SWHS solves this ODE.

\section{General System Description}

This section provides general information about the system.  It identifies the
interfaces between the system and its environment, describes the user
characteristics and lists the system constraints.  

\subsection{System Context}

The system context is shown in Figure \ref{Fig_SystemContext} below. The circles represent the user, who is both responsible for handling the inputs and the outputs. The box represents the program itself, and the arrows indicate what data and information is passed from the user to the program. 

\begin{figure}[h!]
\begin{center}
\includegraphics[width=0.6\textwidth]{SystemContext}
\caption{System Context}
\label{Fig_SystemContext} 
\end{center}
\end{figure}

\begin{itemize}
\item User Responsibilities:
\begin{itemize}
\item Provide required inputs including number of reflectors, glass area, thickness, dimension and size. 
\item Ensure all inputs are in correct format. 
\end{itemize}

\item Responsibilities:
\begin{itemize}
\item Detect data type mismatch, such as a string of characters instead of a
  floating point number
\item Determine if the inputs satisfy the required physical and software constraints such as thickness of the glass can not be a negative value
\item Calculate and plot the required outputs of temperature
\end{itemize}
\end{itemize}

\subsection{User Characteristics} \label{SecUserCharacteristics}

The end user of the system is expected to be familiar with Undergraduate level Calculus and basic physics. They should also know basics about the Reflector angle and heat flows in solar cooker.  

\subsection{System Constraints}

There are no system constraints for this project.

\section{Specific System Description}

This section first presents the problem description, which gives a high-level
view of the problem to be solved.  This is followed by the solution characteristics
specification, which presents the assumptions, theories, definitions and finally
the instance models (ODE) that models the Solar Cooker Reflections.

\subsection{Problem Description} \label{Sec_pd}

Solar Cooker Energy Calculation is intended to investigate the temperature inside the solar cooker box with internal reflectors. 

\subsubsection{Terminology and  Definitions}

This subsection provides a list of terms that are used in the subsequent
sections and their meaning, with the purpose of reducing ambiguity and making it
easier to correctly understand the requirements:

\begin{itemize}

\item Reflectivity: The fraction of radiation reflected by the surface is called the reflectivity ($\rho$) 
\item Transmittivity: The fraction of radiation transmitted is called the transmissivity ($\tau$)
\item Emittance: the energy radiated by the surface of a body per second per unit area ($\epsilon$)
\item Reflactor Angle: the angle between a reflected ray and the normal drawn at the point of incidence to a reflecting surface ($\theta$)
\item Heat flow Convection: Convection is the transfer of heat from one place to another due to the movement of fluid 
\item Heat flow radiation: a process where heat waves are emitted that may be absorbed, reflected, or transmitted through a colder body
\item Heat Convection Coefficients: The rate of heat transfer between a solid surface and a fluid per unit surface area per unit temperature difference (h) 
\item Steffan-Boltzman constant: is a physical constant expressing the relationship between the heat radiation emitted by a black body and its absolute temperature ($\sigma$)
\item Heat Flux: the amount of heat energy passing through a certain surface


\end{itemize}

\subsubsection{Physical System Description} \label{sec_phySystDescrip}

The physical system of Solar Cooker
Energy Calculation, as shown in Figure \ref{Fig_HeatFlows},
includes the following elements:


\begin{itemize}

\item[PS1:] A cover with two flat glasses (glass 1 and glass 2) 

\item[PS2:] Lead of the recipient and recipient itself

\item[PS3:] Fluid inside the recipient

\end{itemize}

\begin{figure}[h!]
\begin{center}
\includegraphics[width=0.45\textwidth]{HeatFlow}
\caption{Heat Flows in Solar Cooker}
\label{Fig_HeatFlows} 
\end{center}
\end{figure}


% \begin{figure}[h!]
% \begin{center}
% %\rotatebox{-90}
% {
%  \includegraphics[width=0.5\textwidth]{<FigureName>}
% }
% \caption{\label{<Label>} <Caption>}
% \end{center}
% \end{figure}

\subsubsection{Goal Statements}

\noindent Given the thermal attributes and solar box characteristics the goal statements are:

\begin{itemize}

\item[GS\refstepcounter{goalnum}\thegoalnum \label{G_meaningfulLabel}:] predicts the Balance of temperature on the recipient.

\item[GS\refstepcounter{goalnum}\thegoalnum \label{G_meaningfulLabel}:] predicts the cooking energy in the recipient over time.

\end{itemize}

\subsection{Solution Characteristics Specification}

The instance model (ODE) that governs Solar Cooker
Energy Calculation is presented in
Subsection~\ref{sec_instance}.  The information to understand the meaning of the
instance model and its derivation is also presented, so that the instance
model can be verified.

\subsubsection{Assumptions} \label{sec_assumpt}


This section simplifies the original problem and helps in developing the
theoretical model by filling in the missing information for the physical
system. The numbers given in the square brackets refer to the theoretical model
[T], general definition [GD], data definition [DD], instance model [IM], or
likely change [LC], in which the respective assumption is used.

\begin{itemize}

\item[A\refstepcounter{assumpnum}\theassumpnum \label{A_common_constant_A_1}:] The emissivity(\si{\epsilon}) have been considered constant [\dref{HFC}, \ddref{dd_q_15}, \ddref{dd_q_16}, \iref{ewat}] 

\item[A\refstepcounter{assumpnum}\theassumpnum \label{A_common_constant_A_2}:] The reflectivity (\si{\rho}) have been considered constant [\ddref{dd_q_14}]

\item[A\refstepcounter{assumpnum}\theassumpnum \label{A_common_constant_A_3}:] The Transmisivity (\si{\tau}) have been considered constant [\ddref{dd_q_14}]

\item[A\refstepcounter{assumpnum}\theassumpnum \label{A_fluid_type}:] The material in the recipient (Container) is liquid for this case (For us, it's water). This implies that the  temperature will not drop below melting point and not rise above the boiling point [\iref{ewat}, \tref{TM_2}]   

\item[A\refstepcounter{assumpnum}\theassumpnum \label{A_int_temp_formula}:] The temperature $T_\text{int2}$ are obtained in function of others temperatures by means of the following supposition: [\iref{ewat}, \ddref{dd_q_13}] 
~\newline


\begin{center}  
$T_\text{int2} = \frac{T_\text{g2} + T_t + T_r}{3} $ 
\end{center}

\item[A\refstepcounter{assumpnum}\theassumpnum \label{A_radiation_impact}:] The solar radiation impact over the solar cooker occurs in perpendicular way. [\iref{ewat}, \ddref{dd_q_14}] 

\item[A\refstepcounter{assumpnum}\theassumpnum \label{A_radiation_impact_a_7}:] The only form of energy considered in this problem is thermal energy. Other energy such as mechanical energy are assumed to be a negligible. [\iref{ewat}, \lcref{LC_1}] 

\item[A\refstepcounter{assumpnum}\theassumpnum \label{A_radiation_impact_a_8}:] The temperature of the reflector is constant over the different time. [\ddref{FluxCoil}, \lcref{LC_2}] 


\end{itemize}

\subsubsection{Theoretical Models}\label{sec_theoretical}

This section focuses on the general equations and laws that Solar Cooker
Energy Calculation is based
on.

~\newline

\noindent
\begin{minipage}{\textwidth}
\renewcommand*{\arraystretch}{1.5}
\begin{tabular}{| p{\colAwidth} | p{\colBwidth}|}
  \hline
  \rowcolor[gray]{0.9}
  Number& TM\refstepcounter{theorynum}\thetheorynum \label{TM_1}\\
  \hline
  Label& \bf Convective Heat Transfer Coefficient\\
  \hline
  Equation &
    h = $\frac{q}{\triangle T}$ \\ 
  \hline
  Description
    & The above equation is used to calculate the heat transfer typically by convection or phase transition.  \\
  
   & $h$ is the heat transfer convection coefficient $(\si[per-mode=symbol] {\watt\per\square\metre} K).$  \\
  
  & $q$ is the thermal flux vector $(\si{\watt\per\square\metre} )$.  \\
  
  & $\triangle$T is the change in temperature. \\
  \hline
  Notes & none. \\
  \hline
  Sources& \url{https://en.wikipedia.org/wiki/Heat_transfer_coefficient} \\
  \hline
  Ref.\ By &  \dref{HFC}, \iref{ewat} \\
  \hline
\end{tabular}
\end{minipage}\\
~\newline

\noindent
\begin{minipage}{\textwidth}
\renewcommand*{\arraystretch}{1.5}
\begin{tabular}{| p{\colAwidth} | p{\colBwidth}|}
  \hline
  \rowcolor[gray]{0.9}
  Number& TM\refstepcounter{theorynum}\thetheorynum \label{TM_2}\\
  \hline
  Label& \bf Sensible Heat energy Calculation\\
  \hline
  Equation &
    Q = $Cm \triangle T$ \\ 
  \hline
  Description
    & This calculation occurs until the highest or lowest temperature reach, as assumed in [\aref{A_fluid_type}] \\
  & $Q$ is the quantity of heat transferred to or from the object \\ 
  & $m$ is the mass of the object \\ 
 & $C$ is the specific heat capacity of the material the object is composed of \\ 
  & $\si{\triangle} T$ is the resulting temperature change of the object   \\
  
  \hline
  Notes & none. \\
  \hline
  Sources& \url{https://www.physicsclassroom.com/class/thermalP/Lesson-2/Measuring-the-Quantity-of-Heat} \\
  \hline
  Ref.\ By &  \iref{I_HETR}\\
  \hline
\end{tabular}
\end{minipage}\\


~\newline


\subsubsection{General Definitions}\label{sec_gendef}


This section collects the laws and equations that will be used in building the
instance models.

~\newline

\noindent
\begin{minipage}{\textwidth}
\renewcommand*{\arraystretch}{1.5}
\begin{tabular}{| p{\colAwidth} | p{\colBwidth}|}
\hline
\rowcolor[gray]{0.9}
Number& GD\refstepcounter{defnum}\thedefnum \label{HFC}\\
\hline
Label &\bf Heat Flow Convection \\
\hline
% Units&$MLt^{-3}T^0$\\
% \hline
SI Units&\si{\watt}\\
\hline
Equation&$ Q(t) = hA \Delta T(t)$  \\
\hline
Description &
Heat Flow Convection is related to Newton's law of cooling describes convective cooling from a surface.  The law is
stated as: the rate of heat loss from a body is proportional to the difference
in temperatures between the body and its surroundings.
\\
& $Q(t)$ is the thermal flux (\si{\watt\per\square\metre}).\\
& $h$ is the heat transfer coefficient
	(\si{\watt\per\square\metre\per\celsius}).\\
 & $A$ is the exposed surface area (\si[per-mode=symbol] {\square\metre}). \\  
&$\Delta T(t)$ is the temperature difference (\si{\celsius}).
\\
\hline
  Source & \url{https://en.wikipedia.org/wiki/Convection_(heat_transfer)} \\
  \hline
  Ref.\ By & \iref{ewat}\\
  \hline
\end{tabular}
\end{minipage}\\


~\newline

\noindent
\begin{minipage}{\textwidth}
\renewcommand*{\arraystretch}{1.5}
\begin{tabular}{| p{\colAwidth} | p{\colBwidth}|}
\hline
\rowcolor[gray]{0.9}
Number& GD\refstepcounter{defnum}\thedefnum \label{HFR}\\
\hline
Label &\bf Heat Flow Radiation \\
\hline
% Units&$MLt^{-3}T^0$\\
% \hline
SI Units&\si{\joule}\\
\hline
Equation&$ Q(t) = \sigma eA \triangle T^4$  \\
\hline
Description &
An object emits radiant energy in all directions unless its temperature is absolute zero. If this energy strikes a receiver, part of it may be absorbed, part may be transmitted, and part may be reflected. Heat transfer from a hot to a cold object in this manner is known as radiation heat transfer . The higher the temperature, the greater is the amount of energy radiated.
\\
& $Q(t)$ is the heat flow radiation (\si{\watt\per\square\metre}).\\
& $\sigma$ is the Steffan-Boltzman constant
	$(5.669 X 10^{-8} W / m^2 K^4)$.\\
 & $e$ is the emissivity of object(\si[per-mode=symbol] {\watt\per\metre}). \\  
 & $A$ is the exposed surface area (\si[per-mode=symbol] {\square\metre}) \\
&$\triangle T(t)$ is the temperature difference (\si{\celsius}).
\\
\hline
  Source & ~\cite{rediationdef} \\
  \hline
  Ref.\ By & \iref{ewat}\\
  \hline
\end{tabular}
\end{minipage}\\


~\newline


\subsubsection{Data Definitions}\label{sec_datadef}


This section collects and defines all the data needed to build the instance
models. The dimension of each quantity is also given.  

~\newline

\noindent
\begin{minipage}{\textwidth}
\renewcommand*{\arraystretch}{1.5}
\begin{tabular}{| p{\colAwidth} | p{\colBwidth}|}
\hline
\rowcolor[gray]{0.9}
Number& DD\refstepcounter{datadefnum}\thedatadefnum \label{dd_q_13}\\
\hline
Label& \bf Data Definition of Q13\\
\hline
Symbol &$Q13$\\
\hline
% Units& $Mt^{-3}$\\
% \hline
  SI Units & \si{\watt\per\square\metre}\\
  \hline
  Equation&$\textbf{Q13} = A_r h_\text{r-int2}(T_\text{int2} - T_r)$ \\
  \hline
  Description & This equation calculates the Heat flow convection of recipient to inner 2 \\
  
  &$A_r$ is the Area of Reflector (\si{\square\metre}).  \\
               &$h_\text{r-int2}$ is the thermal flux difference between Reflector and Inner area inside the box \\ 
                &$T_\text{int2} - T_r$ is the temperature difference between inner area and reflector (\si{\celsius}). 
\\
  \hline
  Sources& ~\cite{MathsModel} \\
  \hline
  Ref.\ By & \iref{ewat}\\
  \hline
\end{tabular} \\
\end{minipage}\\

~\newline


\noindent
\begin{minipage}{\textwidth}
\renewcommand*{\arraystretch}{1.5}
\begin{tabular}{| p{\colAwidth} | p{\colBwidth}|}
\hline
\rowcolor[gray]{0.9}
Number& DD\refstepcounter{datadefnum}\thedatadefnum \label{dd_q_14}\\
\hline
Label& \bf Data Definition of Q14\\
\hline
Symbol &$Q14$\\
\hline
% Units& $Mt^{-3}$\\
% \hline
  SI Units & \si{\watt\per\square\metre}\\
  \hline
  Equation&$\textbf{Q14} = \sum_{i=1}^n \rho A_\text{ref,n} G \tau_g^2 cos (90 - \theta_\text{ref,n})$ \\
  \hline
  Description & This equation calculates the Heat flow reflection of incident radiation on the reflectors \\
  
  &$\rho$ is the reflectivity constant (\si{kg\per\metre^3}).  \\
               &$A_\text{ref,n}$ is the Area of the number of reflectors  \\ 
               &$G$ is Incidental solar radiation taken as an input (\si[per-mode=symbol] {\watt\per\square\metre})  \\ 
               &$\tau_g$ is the Transmittivity constant for glass $(0 \leq \tau \leq 1)$ \\ 
                &$cos(90-\theta_\text{ref,n})$ is the angle difference of reflectors. 
\\
  \hline
  Sources& ~\cite{MathsModel} \\
  \hline
  Ref.\ By & \iref{ewat}\\
  \hline
\end{tabular} \\
\end{minipage}\\

~\newline


\noindent
\begin{minipage}{\textwidth}
\renewcommand*{\arraystretch}{1.5}
\begin{tabular}{| p{\colAwidth} | p{\colBwidth}|}
\hline
\rowcolor[gray]{0.9}
Number& DD\refstepcounter{datadefnum}\thedatadefnum \label{dd_q_15}\\
\hline
Label& \bf Data Definition of Q15\\
\hline
Symbol &$Q15$\\
\hline
% Units& $Mt^{-3}$\\
% \hline
  SI Units & \si{\watt\per\square\metre}\\
  \hline
  Equation&$\textbf{Q15} = A_r \sigma \epsilon_r (T^4_r - T^4_\text{g2})$ \\
  \hline
  Description & This equation calculates the Heat flow radiation of recipient toward glass 2 \\
  
  &$A_r$ is the Area of Reflector (\si{\square\metre}).  \\
               &$\sigma$ is the Steffan-Boltzman constant $(5.669 X 10^{-8} W / m^2 K^4)$ \\ 
               &$\epsilon_r$ is the Emittance of Reflector (\si[per-mode=symbol] {\watt\per\metre})  \\ 
                &$T_r - T_\text{g2}$ is the temperature difference of reflectors and glass 2 (\si{\celsius}). 
\\
  \hline
  Sources& ~\cite{MathsModel} \\
  \hline
  Ref.\ By & \iref{ewat}\\
  \hline
\end{tabular} \\
\end{minipage}\\

~\newline


\noindent
\begin{minipage}{\textwidth}
\renewcommand*{\arraystretch}{1.5}
\begin{tabular}{| p{\colAwidth} | p{\colBwidth}|}
\hline
\rowcolor[gray]{0.9}
Number& DD\refstepcounter{datadefnum}\thedatadefnum \label{dd_q_16}\\
\hline
Label& \bf Data Definition of Q16\\
\hline
Symbol &$Q16$\\
\hline
% Units& $Mt^{-3}$\\
% \hline
  SI Units & \si{\watt\per\square\metre}\\
  \hline
  Equation&$\textbf{Q16} = A_r \sigma \epsilon_r (T^4_r - T^4_f)$ \\
  \hline
  Description & This equation calculates the Heat flow radiation of recipient toward the fluid \\
  
  &$A_r$ is the Area of Reflector (\si{\square\metre}).  \\
               &$\sigma$ is the Steffan-Boltzman constant $(5.669 X 10^{-8} W / m^2 K^4)$ \\ 
               &$\epsilon_r$ is the Emittance of Reflector (\si[per-mode=symbol] {\watt\per\metre})  \\ 
                &$T_r - T_f$ is the temperature difference of reflectors and fluid(\si{\celsius}). 
\\
  \hline
  Sources& ~\cite{MathsModel} \\
  \hline
  Ref.\ By & \iref{ewat}\\
  \hline
\end{tabular} \\
\end{minipage}\\

~\newline

\noindent
\begin{minipage}{\textwidth}
\renewcommand*{\arraystretch}{1.5}
\begin{tabular}{| p{\colAwidth} | p{\colBwidth}|}
\hline
\rowcolor[gray]{0.9}
Number& DD\refstepcounter{datadefnum}\thedatadefnum \label{dd_q_17}\\
\hline
Label& \bf Data Definition of Q17\\
\hline
Symbol &$Q17$\\
\hline
% Units& $Mt^{-3}$\\
% \hline
  SI Units & \si{\watt\per\square\metre}\\
  \hline
  Equation&$\textbf{Q17} = A_m h_\text{r-f}(T_r - T_f)$ \\
  \hline
  Description & This equation calculates the Heat flow convection of recipient toward the fluid \\
  
  &$A_m$ is the wet inside the container (\si{\square\metre}).  \\
               &$h_\text{r-f}$ is the thermal flux difference between Reflector and fluid inside the container \\ 
                &$T_r - T_f$ is the temperature difference between reflector and fluid (\si{\celsius}). 
\\
  \hline
  Sources& ~\cite{MathsModel} \\
  \hline
  Ref.\ By & \iref{ewat}\\
  \hline
\end{tabular} \\
\end{minipage}\\



~\newline

\noindent
\begin{minipage}{\textwidth}
\renewcommand*{\arraystretch}{1.5}
\begin{tabular}{| p{\colAwidth} | p{\colBwidth}|}
\hline
\rowcolor[gray]{0.9}
Number& DD\refstepcounter{datadefnum}\thedatadefnum \label{FluxCoil}\\
\hline
Label& \bf Heat flux over all different object\\
\hline
Symbol &$q$\\
\hline
% Units& $Mt^{-3}$\\
% \hline
  SI Units & \si{\watt\per\square\metre}\\
  \hline
  Equation&$q (t) = h_r (T_r - T_W (t) )$, over area $A_r$ \\
  \hline
  Description & It is necessary to know the thermal conductivity of a material if you want to calculate the heat energy transferred through it \\
  
  &$q$ is the heat flux  \\
               &$h_r$ is the convective heat transfer coefficient between reflector and water \\ 
                &$T_r$ is the temperature of reflector \si{\celsius} \\
                &$T_W(t)$ is the temperature of water \si{\celsius} \\
                &$t$ is the time (s) 
\\
  \hline
  Sources& \url{https://www.omnicalculator.com/physics/thermal-conductivity} \\
  \hline
  Ref.\ By & \iref{ewat}\\
  \hline
\end{tabular} \\
\end{minipage}\\


\subsubsection{Instance Models} \label{sec_instance}    

This section transforms the problem defined in Section~\ref{Sec_pd} into 
one which is expressed in mathematical terms. It uses concrete symbols defined 
in Section~\ref{sec_datadef} to replace the abstract symbols in the models 
identified in Sections~\ref{sec_theoretical}.


~\newline

%Instance Model 1

\noindent
\begin{minipage}{\textwidth}
\renewcommand*{\arraystretch}{1.5}
\begin{tabular}{| p{\colAwidth} | p{\colBwidth}|}
  \hline
  \rowcolor[gray]{0.9}
  Number& IM\refstepcounter{instnum}\theinstnum \label{ewat}\\
  \hline
  Label& \bf Balance of energy on the recipient to find $T_W$\\
  \hline
  Input&$A_r$, $\epsilon_r$, $T_r$, $T_f$, $T_\text{int2}$, $A_m$, $T_f$, $G$\\
  & The input is constrained so that $\epsilon_t \leq 0 $ \\
  \hline
  Output&$T_f(t)$, $t \geq 0 $, such that\\

  & $ \bf m_W c_W  \frac{dT_W}{dt}  = Q13 + 4Q14 - Q15 - Q16 - Q17 $\\
  \hline
  Description
  &$A_r$ is the Area of Reflector (\si{\square\metre}).\\
  &$\epsilon_r$ is the Emittance of recipient (\si[per-mode=symbol] {\watt\per\metre}).\\
  &$T_r$ is the Reflector temperature (\si{\celsius}).\\
  &$T_\text{int2}$ is a temperature inside the box (\si{\celsius}).\\
  &$G$ is the Incidental solar radiation (\si[per-mode=symbol] {\watt\per\square\metre}).\\
  &$T_f$ is a current temperature of Fluid (\si{\celsius}).\\
  

  & The above equation applies as long as the fluid is in liquid form,
  $0<T_f<100^o\text{C}$, where $0^o\text{C}$ and $100^o\text{C}$ are the melting
  and boiling points of water(as we are using water for testing), respectively.
  \\
  \hline
  Sources& \cite{MathsModel} \\
  \hline
  Ref.\ By & None\\
  \hline
\end{tabular}
\end{minipage}\\

~\newline

\noindent
\begin{minipage}{\textwidth}
\renewcommand*{\arraystretch}{1.5}
\begin{tabular}{| p{\colAwidth} | p{\colBwidth}|}
  \hline
  \rowcolor[gray]{0.9}
  Number& IM\refstepcounter{instnum}\theinstnum \label{I_HETR}\\
  \hline
  Label& \bf Heat energy in the recipient\\
  \hline
  Input&$C_W$, $m_W$, $T_\text{init}$, $T_W(t)$\\
  \hline
  Output&$E_W(t)$, $0 \leq t \leq t_\text{final}$, such that\\
  &$E_W(t)$ = $C_W m_W (T_W(t) - T_\text{init})$\\
  \hline
  Description & The above equation is derived using \tref{T:SHE}.  $E_W$ is the 
  change in thermal energy of the liquid water relative to the energy at the initial 
  temperature ($T_\text{init}$).  $C_W$ is the specific heat capacity of liquid water and $m_W$ is 
  the mass of the water.  The change in temperature is the difference between 
  the temperature at time t, $T_W$, and the initial temperature, $T_\text{init}$, this
  equation applies as long as $0 < T_W < 100^o\text{C}$ (\aref{A_fluid_type}).\\
  \hline
  Sources&~\cite{cookingpower}\ \\
  \hline
  Ref.\ By & None\\
  \hline
\end{tabular}
\end{minipage}\\

~\newline


\subsubsection{Input Data Constraints} \label{sec_DataConstraints}    

Table~\ref{TblInputVar} shows the data constraints on the input output
variables.  The column for physical constraints gives the physical limitations
on the range of values that can be taken by the variable.  The column for
software constraints restricts the range of inputs to reasonable values.  The
software constraints will be helpful in the design stage for picking suitable
algorithms.  The constraints are conservative, to give the user of the model the
flexibility to experiment with unusual situations.  The column of typical values
is intended to provide a feel for a common scenario.  The uncertainty column
provides an estimate of the confidence with which the physical quantities can be
measured.  This information would be part of the input if one were performing an
uncertainty quantification exercise.


\begin{table}[!h]
  \caption{Input Variables} \label{TblInputVar}
  \renewcommand{\arraystretch}{1.2}
\noindent \begin{longtable*}{l l l l c} 
  \toprule
  \textbf{Var} & \textbf{Physical Constraints} & \textbf{Software Constraints} &
                             \textbf{Typical Value} & \textbf{Uncertainty}\\
  \midrule 
  $L$ & $L > 0$ & $L_{\text{min}} \leq L \leq L_{\text{max}}$ & 0.12 \si[per-mode=symbol] {\metre} & -
  \\
  $C_w$ & $C_w > 0$ & $C_w^\text{min} \leq C_w \leq C_w^\text{max}$ & 4186 \si[per-mode=symbol] {\joule\per\ (kg \celsius})    & - 
  \\
  $\tau_g$ & $0 \leq \tau_g \leq 1$ & - & 0.48 \si[per-mode=symbol] {\metre} & - 
  \\
  $A_g$ & $A_g > 0$ & $A_g \leq A_g^\text{max}$ & 0.24 \si[per-mode=symbol] {\square\metre}   & - 
  \\
  $n$ & $n > 0$ & $n_\text{min} \leq n \leq n_\text{max} $ & 1  & - 
  \\
  $T_\text{init}$ & $0 < T_\text{init} < 100$ & - & 40 \si[per-mode=symbol] {\celsius} & - 
  \\
  $G$ & $G > 0$ & $G_{\text{min}} \leq L \leq G_{\text{max}}$ & 476 \si[per-mode=symbol] {\watt\per\square\metre} & - 
  \\
  $T_\text{f}$ & $0 < T_\text{f} < 100$ & - & 40 \si[per-mode=symbol] {\celsius} & - 
  \\
  
  \bottomrule
\end{longtable*}
\end{table}


\begin{table}[!h]
\caption{Output Variables} \label{TblOutputVar}
\renewcommand{\arraystretch}{1.2}
\noindent \begin{longtable*}{l l} 
  \toprule
  \textbf{Var} & \textbf{Physical Constraints} \\
  \midrule 
  $T_r$ & $T_\text{init} \leq T_r $
  \\
  $E_w$ & $E_w > 0$ \\
  \bottomrule
\end{longtable*}
\end{table}


\section{Requirements}


This section provides the functional requirements, the business tasks that the
software is expected to complete, and the nonfunctional requirements, the
qualities that the software is expected to exhibit.

\subsection{Functional Requirements}

\noindent \begin{itemize}

\item[R\refstepcounter{reqnum}\thereqnum \label{R_Inputs}:] Provide the inputs specific to Solar Box (Area of glass, Current Temperature of reflector/fluid/glass, Area of reflector ) and other parameters (Incidental Solar Radiation, Reflectors angle, and number of reflectors)

\item[R\refstepcounter{reqnum}\thereqnum \label{R_VerifyOutput}:]
  Verify the given inputs are in correct form and satify the required physical constraints given in the table~\ref{TblInputVar}

\item[R\refstepcounter{reqnum}\thereqnum \label{R_Calculate}:] Calculate and output the balance of energy on recipient $T_W(t)$ over the simulation time(from \iref{ewat})

\item[R\refstepcounter{reqnum}\thereqnum \label{R_Output}:] Calculate and output the energy in the water (fluid - recipient) $E_W(t)$ over the simulation time(from \iref{I_HETR})

\end{itemize}


\subsection{Nonfunctional Requirements}


\noindent 

This problem is small in size and relatively simple, so performance is not a priority. Any reasonable implementation will be very quick and use minimal storage. Rather than performance, the non-functional requirement priorities are correctness, verifiability, understandability, reusability, and maintainability. 
 


\section{Likely Changes}    

\noindent \begin{itemize}

\item[LC\refstepcounter{lcnum}\thelcnum\label{LC_1}:] \aref{A_radiation_impact_a_7} 
The other form of energy such as Mechanical energy may apply as a part og energy consideration.

\item[LC\refstepcounter{lcnum}\thelcnum\label{LC_2}:] \aref{A_radiation_impact_a_8} 
The temperature of the reflector will change over the different time during the day which is totally based on the sun rays. 

\end{itemize}

\section{Unlikely Changes}    

\noindent \begin{itemize}

\item[LC\refstepcounter{lcnum}\thelcnum\label{LC_3}:] 
\aref{A_int_temp_formula} 
The given calculation for the inner temperature for the box(int2) in the assumption will not change throughout the whole process. 

\end{itemize}

\section{Traceability Matrices and Graphs}

The purpose of the traceability matrices is to provide easy references on what
has to be additionally modified if a certain component is changed.  Every time a
component is changed, the items in the column of that component that are marked
with an ``X'' may have to be modified as well.  Table~\ref{Table:trace} shows the
dependencies of theoretical models, general definitions, data definitions, and
instance models with each other. Table~\ref{Table:R_trace} shows the
dependencies of instance models, requirements, and data constraints on each
other. Table~\ref{Table:A_trace} shows the dependencies of theoretical models,
general definitions, data definitions, instance models, and likely changes on
the assumptions.

~\newline

\begin{table}[h!]
\centering
\begin{tabular}{|c|c|c|c|c|c|c|c|}
\hline
	& \iref{ewat}& \iref{I_HETR}& \ref{sec_DataConstraints}& \rref{R_Inputs}& \rref{R_VerifyOutput} & \rref{R_Calculate} & \rref{R_Output} \\
\hline
\iref{ewat}            & & X& & & & X&  \\ \hline
\iref{I_HETR}            & X& & & & & & X \\ \hline
\rref{R_Inputs}     & & & & & & & \\ \hline
\rref{R_VerifyOutput}  & & & X& & & & \\ \hline
\rref{R_Calculate}    &X & & & & & & \\ \hline
\rref{R_Output}  & & X& & & & &  \\ 
\hline
\end{tabular}
\caption{Traceability Matrix Showing the Connections Between Requirements and Instance Models}
\label{Table:R_trace}
\end{table}



\begin{table}[h!]
\centering
\begin{tabular}{|c|c|c|c|c|c|c|c|c|}
\hline
	& \aref{A_common_constant_A_1}& \aref{A_common_constant_A_2}& \aref{A_common_constant_A_3}& \aref{A_fluid_type}& \aref{A_int_temp_formula}& \aref{A_radiation_impact}& \aref{A_radiation_impact_a_7}& \aref{A_radiation_impact_a_8} \\

 

\hline
\tref{TM_1}     & X& & & & & & &  \\ \hline
\tref{TM_2}     & X& & & & & & & \\ \hline
\dref{HFC}        & X& & & & & & &  \\ \hline
\dref{HFR}      & X& & & & & & &  \\ \hline
\ddref{dd_q_13} & X& & & & & & &  \\ \hline
\ddref{dd_q_14}  & X& & & & & & &  \\ \hline
\ddref{dd_q_15}    & X& & & & & & &  \\ \hline
\ddref{dd_q_16}     & X& & & & & & & \\ \hline
\ddref{dd_q_17}     & X& & & & & & &  \\ \hline
\ddref{FluxCoil}     & X& & & & & & &  \\ \hline
\iref{ewat}      & X& & & & & & &  \\ \hline
\iref{I_HETR}      & X& & & & & & &  \\ \hline
\lcref{LC_1}      & X& & & & & & &  \\ \hline
\lcref{LC_2}      & X& & & & & & &  \\ 

\hline
\end{tabular}
\caption{Traceability Matrix Showing the Connections Between Assumptions and Other Items}
\label{Table:A_trace}
~\newline
\end{table}



\begin{table}[h!]
\centering
\begin{tabular}{|c|c|c|c|c|c|c|c|c|c|c|c|c|c|c|c|c|c|c|c|c|c|c|c|}
\hline        
	& \tref{TM_1}& \tref{TM_2}& \dref{HFC}& \dref{HFR} & \ddref{dd_q_13}& \ddref{dd_q_14} & \ddref{dd_q_15}& \ddref{dd_q_16}& \ddref{dd_q_17}& \ddref{FluxCoil} &\iref{ewat}& \iref{I_HETR} \\
\hline
\tref{TM_1}     & & & X & & X& & & & & & X &  \\ \hline
\tref{TM_2}     & & & & & & & & & & & & X \\ \hline
\dref{HFC}        &X & & & & X& & & & X& & &  \\ \hline
\dref{HFR}      & & & & & & & X& X& & & &  \\ \hline
\ddref{dd_q_13} & X& & X& & & & & & & & X&  \\ \hline
\ddref{dd_q_14}  & & & & & & & & & & & X&  \\ \hline
\ddref{dd_q_15}    & & & &X & & & & & & & X&  \\ \hline
\ddref{dd_q_16}     & & & &X & & & & & & &X & \\ \hline
\ddref{dd_q_17}     &X & &X & & & & & & & & X&  \\ \hline
\ddref{FluxCoil}     &X & & & & & & & & & & X&  \\ \hline
\iref{ewat}      &X & & & &X &X &X &X &X &X & &  \\ \hline
\iref{I_HETR}      & & X& & & & & & & & & &  \\ 
\hline
\end{tabular}
\caption{Traceability Matrix Showing the Connections Between Items of Different Sections}
\label{Table:trace}
\end{table} 

~\newline


\newpage

\bibliographystyle {plain}
\bibliography {srs}

\newpage

\end{document}
