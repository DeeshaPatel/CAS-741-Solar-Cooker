\documentclass{article}

\usepackage{amsmath, mathtools}
\usepackage{amsfonts}
\usepackage{amssymb}
\usepackage{graphicx}
\usepackage{colortbl}
\usepackage{xr}
\usepackage{hyperref}
\usepackage{longtable}
\usepackage{xfrac}
\usepackage{tabularx}
\usepackage{float}
\usepackage{siunitx}
\usepackage{booktabs}
\usepackage{caption}
\usepackage{pdflscape}
\usepackage{afterpage}

\usepackage[round]{natbib}

\hypersetup{       
    colorlinks=true,       % false: boxed links; true: colored links
    linkcolor=red,          % color of internal links (change box color with linkbordercolor)
    citecolor=green,        % color of links to bibliography
    filecolor=magenta,      % color of file links
    urlcolor=cyan           % color of external links
}

\title{Problem Statement and Goals\\Solar Cooker Energy Calculation}

\author{Deesha Patel}

\date{January 20th, 2023}

\input{}
\input{}

\begin{document}

\maketitle
\begin{table}[hp]
\caption{Revision History} \label{TblRevisionHistory}
\begin{tabularx}{\textwidth}{llX}
\toprule
\textbf{Date} & \textbf{Developer(s)} & \textbf{Change}\\
\midrule
20 January 2023 & Deesha Patel & Initial release of document\\
22 January 2023 & Deesha Patel & Updates according to issue\\ 
\bottomrule
\end{tabularx}
\end{table}

\section{Problem Statement}

Renewable energy is essential nowadays as fossil fuel causing damage to the environment in all ways. Many countries have started to develop the renewable energy. Solar energy can be used in heating water, cooking food and transforming it into another form of energy. 


\subsection{Problem}

 The main challenge for developing is how to utilize more solar energy by changing the box design or choice of glass. One work by Grupp, M., Montagne, P., and Wackernagel, M.~\cite{NovelBoxType} includes replacing the conventional solar cooker (where we can put the pot anywhere in the box) with the fixed pot position in a solar cooker. Even though such design choices give good results, we need to focus on how the internal reflector of the glasses can affect the cooking temperature in the conventional box. This project aims to calculate and analyze how internal reflectors can be helpful in improving the overall cooking temperature in the solar cooker. We got the inspiration for this project from a proposed mathematical model of internal reflectors~\cite{MathsModel}. And when the internal steps are added, the inside temperatures in the solar cooker also increase. 

\subsection{Inputs and Outputs}

It will take several inputs including the number of reflectors, glass considerations (angle, size, dimension, thickness), and solar radiation, and give an output of our analysis to use the internal reflector method in increasing temperature in the box. 

\subsection{Stakeholders}

The companies who manufacture the solar cooker are the potential stakeholders of our system. Citizen of the countries who want to support the country in development of Renewable energy are end-user. 

\subsection{Environment}

This software can be deploy on any operating system including Mac OS, Windows 10 and greater and Linux operating system.   

\section{Goals}
\begin{enumerate}
    \item It will provide improved temperature in Solar Cooker. 
    \item It will give solution to utilize more heat energy for cooking food compare to existing solutions.   
\end{enumerate} 

\section{Stretch Goals}
\begin{enumerate}
    \item We will Visualize the final Structure and make it real. 
    \item We will experiment proposed solution in different environment and regions which can give us a surety that it work as desired. 
\end{enumerate} 

\bibliographystyle {plain}
\bibliography {srs}

\end{document}
