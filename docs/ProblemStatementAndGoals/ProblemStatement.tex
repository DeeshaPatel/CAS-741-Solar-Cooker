\documentclass{article}

\usepackage{tabularx}
\usepackage{booktabs}

\title{Problem Statement and Goals\\\progname}

\author{\authname}

\date{}

%% Comments

\usepackage{color}

\newif\ifcomments\commentstrue %displays comments
%\newif\ifcomments\commentsfalse %so that comments do not display

\ifcomments
\newcommand{\authornote}[3]{\textcolor{#1}{[#3 ---#2]}}
\newcommand{\todo}[1]{\textcolor{red}{[TODO: #1]}}
\else
\newcommand{\authornote}[3]{}
\newcommand{\todo}[1]{}
\fi

\newcommand{\wss}[1]{\authornote{blue}{SS}{#1}} 
\newcommand{\plt}[1]{\authornote{magenta}{TPLT}{#1}} %For explanation of the template
\newcommand{\an}[1]{\authornote{cyan}{Author}{#1}}

%% Common Parts

\newcommand{\progname}{ProgName} % PUT YOUR PROGRAM NAME HERE
\newcommand{\authname}{Team \#, Team Name
\\ Student 1 name
\\ Student 2 name
\\ Student 3 name
\\ Student 4 name} % AUTHOR NAMES                  

\usepackage{hyperref}
    \hypersetup{colorlinks=true, linkcolor=blue, citecolor=blue, filecolor=blue,
                urlcolor=blue, unicode=false}
    \urlstyle{same}
                                


\begin{document}

\maketitle

\begin{table}[hp]
\caption{Revision History} \label{TblRevisionHistory}
\begin{tabularx}{\textwidth}{llX}
\toprule
\textbf{Date} & \textbf{Developer(s)} & \textbf{Change}\\
\midrule
Date1 & Name(s) & Description of changes\\
Date2 & Name(s) & Description of changes\\
... & ... & ...\\
\bottomrule
\end{tabularx}
\end{table}

\section{Problem Statement}

\wss{You should check your problem statement with the
\href{https://github.com/smiths/capTemplate/blob/main/docs/Checklists/ProbState-Checklist.pdf}
{problem statement checklist}.}
\wss{You can change the section headings, as long as you include the required information.}

\subsection{Problem}

\subsection{Inputs and Outputs}

\wss{Characterize the problem in terms of ``high level'' inputs and outputs.  
Use abstraction so that you can avoid details.}

\subsection{Stakeholders}

\subsection{Environment}

\wss{Hardware and software}

\section{Goals}

\section{Stretch Goals}

\end{document}\documentclass{article}

\usepackage{tabularx}
\usepackage{booktabs}

\title{Problem Statement and Goals\\Solar Cooker Energy Calculation}

\author{Deesha Patel}

\date{January 20th, 2023}

\input{}
\input{}

\begin{document}

\maketitle
\begin{table}[hp]
\caption{Revision History} \label{TblRevisionHistory}
\begin{tabularx}{\textwidth}{llX}
\toprule
\textbf{Date} & \textbf{Developer(s)} & \textbf{Change}\\
\midrule
20 January 2023 & Deesha Patel & Initial release of document\\

\bottomrule
\end{tabularx}
\end{table}

\section{Problem Statement}

Renewable energy is essential nowadays as fossil fuel causing damage to the environment in all ways. Many countries have started to develop the renewable energy. Solar energy can be used in heating water, cooking food and transforming it into another form of energy. 


\subsection{Problem}

 The main challenge for developing is that how we can utilise more solar energy especially for cooking meal. Solar Cooker Energy Calculation is developed to calculate and analyse the total used energy and heat loss with all sides in solar cooker box that can help to utilize more heat power.  

\subsection{Inputs and Outputs}

It will take dimensional inputs and temperature parameters and give output of total prediction of energy consumption. 

\subsection{Stakeholders}

The companies who manufacture the solar cooker are the potential stakeholders of our system. Citizen of the countries who want to support the country in development of Renewable energy are end-user. 

\subsection{Environment}

This software can be deploy on Mac OS, Windows 10 and greater and Linux operating system.   

\section{Goals}
\begin{enumerate}
    \item It will provide simplified energy calculation for Solar Cooker. 
    \item It will give solution to utilize more heat energy for cooking food.   
\end{enumerate} 

\section{Stretch Goals}
\begin{enumerate}
    \item We will Visualize the final Structure and make it real. 
    \item We will Test product in different environment and regions. 
\end{enumerate} 

\end{document}
