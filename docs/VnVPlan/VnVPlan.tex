\documentclass[12pt, titlepage]{article}

\usepackage{booktabs}
\usepackage{tabularx}
\usepackage{colortbl}
\usepackage{hyperref}
\hypersetup{
    colorlinks,
    citecolor=blue,
    filecolor=black,
    linkcolor=red,
    urlcolor=blue
}
\usepackage[round]{natbib}


\begin{document}

\title{System Verification and Validation Plan for \progname{Solar Cooker Energy Calculator}} 
\author{Deesha Patel}
\date{\today}
	
\maketitle

\pagenumbering{roman}

\section{Revision History}

\begin{tabularx}{\textwidth}{p{5cm}p{1.5cm}X}
\toprule {\bf Date} & {\bf Version} & {\bf Notes}\\
\midrule
February 14, 2023 & 1.0 & Add General Information section \\
& 1.1 & Add further details in different section \\

\bottomrule
\end{tabularx}

\newpage

\tableofcontents

\listoftables

\listoffigures
\wss{Remove this section if it isn't needed}

\newpage

\section{Symbols, Abbreviations and Acronyms}

\renewcommand{\arraystretch}{1.2}
\begin{tabular}{l l} 
  \toprule		
  \textbf{symbol} & \textbf{description}\\
  \midrule 
  MG & Module Guide \\
  MIS & Module Interface Specification \\
  SRS & Software Requirement Specification\\
  SCEC & Solar Cooker Energy Calculator \\ 
  VnV & Verification and Validation \\ 
  
  \bottomrule
\end{tabular}\\

For complete symbols used within the system, please refer the section 1 in 
  \href{https://github.com/DeeshaPatel/CAS-741-Solar-Cooker/blob/7c53c8d9a19ca2f94dfba6ba9208eae0bf03b8cc/docs/SRS/SRS.pdf}{SRS} document.


\newpage

\pagenumbering{arabic}

This document provides the road-map of the Verification and Validation plan for Solar Cooker Energy Calculator for ensuring the requirements and goals of the program (found in \href{https://github.com/DeeshaPatel/CAS-741-Solar-Cooker/blob/7c53c8d9a19ca2f94dfba6ba9208eae0bf03b8cc/docs/SRS/SRS.pdf}{SRS} document). The organization of this document starts with the General Information about the Solar Cooker Energy Calculator in \autoref{GeneralInformation}. A verification plan is provided in \autoref{plan} and \autoref{systemtests} describes the system tests, including tests for functional and non-functional requirements.
  

\section{General Information}
\label{GeneralInformation}

\subsection{Summary}

This document reviews the validation and verification plan for Solar Cooker Energy Calculator (SCEC), a program that calculate the balance temperature at recipient and cooking power in it using user inputs.  

\subsection{Objectives}

The purpose of the validation plan is to define how system validation will perform at the end of the project - the strategy will use to assess whether the developed system accomplishes the designed goals. Also, the verification plan includes test strategies, definitions of what will be tested, and a test matrix with detailed mapping connecting the testing performed to the system requirements. This verification plan ensures that all requirements specified in the System Requirements Specification(\href{https://github.com/DeeshaPatel/CAS-741-Solar-Cooker/blob/7c53c8d9a19ca2f94dfba6ba9208eae0bf03b8cc/docs/SRS/SRS.pdf}{SRS}) document have been met and reviewed. The specific goal of this document is to demonstrate the adequate usability of the system.


\subsection{Relevant Documentation}

The relevant documentation for the SCEC includes \href{https://github.com/DeeshaPatel/CAS-741-Solar-Cooker/blob/7c53c8d9a19ca2f94dfba6ba9208eae0bf03b8cc/docs/ProblemStatementAndGoals/ProblemStatement.pdf}{Problem Statement}, \href{https://github.com/DeeshaPatel/CAS-741-Solar-Cooker/blob/7c53c8d9a19ca2f94dfba6ba9208eae0bf03b8cc/docs/SRS/SRS.pdf}{System Requirements Specifications}, VnV Report, MG and MIS (found in \href{https://github.com/DeeshaPatel/CAS-741-Solar-Cooker/tree/main/docs}{Github Repository})  


\section{Plan}
\label{plan}

This section describes the plan for the Solar Cooker Energy Calculator system. The planning starts with the Verification and Validation team, followed by the SRS verification plan, Design verification plan, Implementation verification plan, Automated testing and verification tools, and Software validation plan.  

\subsection{Verification and Validation Team}

This section describes the members of Verification and Validation plan. 

\begin{center}
\begin{table}[h!]
\begin{tabular}{ |l|l|p{2cm}|p{5cm}| } 
\hline
\rowcolor[gray]{0.9}
\textbf{Name} & \textbf{Document} & \textbf{Role} & \textbf{Description} \\
\hline
 Dr. Spencer Smith & all & Instructor/ Reviewer & Review the documents, design and documentation style. \\ 
 \hline
 Deesha Patel & all & Author & Create all the documents, provide the VnV plan, test case and test execution, verify the implementation. \\  
 \hline
 Mina Mahdipour & all & Domain Expert Reviewer & Review all the documents and review the VnV plan. \\  
 \hline
 Karen Wang & SRS & Secondary Reviewer & Review the SRS document \\
 \hline
Lesley  Wheat & VnV Plan & Secondary Reviewer & Review the VnV plan. \\ 
\hline 
Sam Joseph Crawford & MG + MIS & Secondary Reviewer & Review the MG and MIS document. \\

\hline
\end{tabular}
\caption{Verification and Validation team} 
\end{table}
\end{center}

\subsection{SRS Verification Plan}

The SCEC SRS document shall be verified in the following way: 

\begin{enumerate}

\item Initial review from the assigned members (Dr. Spencer Smith, Mina Mahdipour, Karen Wang, and Deesha Patel) will be performed. For this, the manual review will perform using the given \href{https://github.com/smiths/capTemplate/blob/9251702fdcb9800c59f6ed3d11d91e2bd62fca6d/docs/Checklists/SRS-Checklist.pdf}{SRS Checklist}, designed by Dr. Smith. 

\item Reviewer can give feedback to the author by creating the issue in Github. 

\item Author (Deesha Patel) is responsible to address the issues created by the primary and secondary reviewers. Also, need to address the suggestions given by the instructor (Dr. Spencer Smith).  

\end{enumerate}


\subsection{Design Verification Plan}

The design documents, Module Guide (MG), and Module Interface Specification (MIS) will be verified through the static technic of document inspection by the Domain/ Primary expert (Mina Mahdipour) and Secondary Reviewer (Sam Joseph Crawford). Also, the class instructor (Dr. Spencer Smith) will review both documents. Reviewers can give feedback to the author by creating the issue in Github. The author is responsible to solve the issues and address the suggestions. The reviewer will assess this document with the help of \href{https://github.com/smiths/capTemplate/blob/9251702fdcb9800c59f6ed3d11d91e2bd62fca6d/docs/Checklists/MG-Checklist.pdf}{MG Checklist} and \href{https://github.com/smiths/capTemplate/blob/9251702fdcb9800c59f6ed3d11d91e2bd62fca6d/docs/Checklists/MIS-Checklist.pdf}{MIS Checklist} designed by Dr. Smith.      

\subsection{Implementation Verification Plan}

\wss{You should at least point to the tests listed in this document and the unit
  testing plan.}

\wss{In this section you would also give any details of any plans for static verification of
  the implementation.  Potential techniques include code walkthroughs, code
  inspection, static analyzers, etc.}

\subsection{Automated Testing and Verification Tools}

\wss{What tools are you using for automated testing.  Likely a unit testing
  framework and maybe a profiling tool, like ValGrind.  Other possible tools
  include a static analyzer, make, continuous integration tools, test coverage
  tools, etc.  Explain your plans for summarizing code coverage metrics.
  Linters are another important class of tools.  For the programming language
  you select, you should look at the available linters.  There may also be tools
  that verify that coding standards have been respected, like flake9 for
  Python.}

\wss{If you have already done this in the development plan, you can point to
that document.}

\wss{The details of this section will likely evolve as you get closer to the
  implementation.}

\subsection{Software Validation Plan}

\wss{If there is any external data that can be used for validation, you should
  point to it here.  If there are no plans for validation, you should state that
  here.}

\wss{You might want to use review sessions with the stakeholder to check that
the requirements document captures the right requirements.  Maybe task based
inspection?}

\wss{This section might reference back to the SRS verification section.}

\section{System Test Description}
\label{systemtests}
	
\subsection{Tests for Functional Requirements}

\wss{Subsets of the tests may be in related, so this section is divided into
  different areas.  If there are no identifiable subsets for the tests, this
  level of document structure can be removed.}

\wss{Include a blurb here to explain why the subsections below
  cover the requirements.  References to the SRS would be good here.}

\subsubsection{Area of Testing1}

\wss{It would be nice to have a blurb here to explain why the subsections below
  cover the requirements.  References to the SRS would be good here.  If a section
  covers tests for input constraints, you should reference the data constraints
  table in the SRS.}
		
\paragraph{Title for Test}

\begin{enumerate}

\item{test-id1\\}

Control: Manual versus Automatic
					
Initial State: 
					
Input: 
					
Output: \wss{The expected result for the given inputs}

Test Case Derivation: \wss{Justify the expected value given in the Output field}
					
How test will be performed: 
					
\item{test-id2\\}

Control: Manual versus Automatic
					
Initial State: 
					
Input: 
					
Output: \wss{The expected result for the given inputs}

Test Case Derivation: \wss{Justify the expected value given in the Output field}

How test will be performed: 

\end{enumerate}

\subsubsection{Area of Testing2}

...

\subsection{Tests for Nonfunctional Requirements}

\wss{The nonfunctional requirements for accuracy will likely just reference the
  appropriate functional tests from above.  The test cases should mention
  reporting the relative error for these tests.  Not all projects will
  necessarily have nonfunctional requirements related to accuracy}

\wss{Tests related to usability could include conducting a usability test and
  survey.  The survey will be in the Appendix.}

\wss{Static tests, review, inspections, and walkthroughs, will not follow the
format for the tests given below.}

\subsubsection{Area of Testing1}
		
\paragraph{Title for Test}

\begin{enumerate}

\item{test-id1\\}

Type: Functional, Dynamic, Manual, Static etc.
					
Initial State: 
					
Input/Condition: 
					
Output/Result: 
					
How test will be performed: 
					
\item{test-id2\\}

Type: Functional, Dynamic, Manual, Static etc.
					
Initial State: 
					
Input: 
					
Output: 
					
How test will be performed: 

\end{enumerate}

\subsubsection{Area of Testing2}

...

\subsection{Traceability Between Test Cases and Requirements}

\wss{Provide a table that shows which test cases are supporting which
  requirements.}

\section{Unit Test Description}

\wss{Reference your MIS (detailed design document) and explain your overall
  philosophy for test case selection.}  
\wss{This section should not be filled in until after the MIS (detailed design
  document) has been completed.}

\subsection{Unit Testing Scope}

\wss{What modules are outside of the scope.  If there are modules that are
  developed by someone else, then you would say here if you aren't planning on
  verifying them.  There may also be modules that are part of your software, but
  have a lower priority for verification than others.  If this is the case,
  explain your rationale for the ranking of module importance.}

\subsection{Tests for Functional Requirements}

\wss{Most of the verification will be through automated unit testing.  If
  appropriate specific modules can be verified by a non-testing based
  technique.  That can also be documented in this section.}

\subsubsection{Module 1}

\wss{Include a blurb here to explain why the subsections below cover the module.
  References to the MIS would be good.  You will want tests from a black box
  perspective and from a white box perspective.  Explain to the reader how the
  tests were selected.}

\begin{enumerate}

\item{test-id1\\}

Type: \wss{Functional, Dynamic, Manual, Automatic, Static etc. Most will
  be automatic}
					
Initial State: 
					
Input: 
					
Output: \wss{The expected result for the given inputs}

Test Case Derivation: \wss{Justify the expected value given in the Output field}

How test will be performed: 
					
\item{test-id2\\}

Type: \wss{Functional, Dynamic, Manual, Automatic, Static etc. Most will
  be automatic}
					
Initial State: 
					
Input: 
					
Output: \wss{The expected result for the given inputs}

Test Case Derivation: \wss{Justify the expected value given in the Output field}

How test will be performed: 

\item{...\\}
    
\end{enumerate}

\subsubsection{Module 2}

...

\subsection{Tests for Nonfunctional Requirements}

\wss{If there is a module that needs to be independently assessed for
  performance, those test cases can go here.  In some projects, planning for
  nonfunctional tests of units will not be that relevant.}

\wss{These tests may involve collecting performance data from previously
  mentioned functional tests.}

\subsubsection{Module ?}
		
\begin{enumerate}

\item{test-id1\\}

Type: \wss{Functional, Dynamic, Manual, Automatic, Static etc. Most will
  be automatic}
					
Initial State: 
					
Input/Condition: 
					
Output/Result: 
					
How test will be performed: 
					
\item{test-id2\\}

Type: Functional, Dynamic, Manual, Static etc.
					
Initial State: 
					
Input: 
					
Output: 
					
How test will be performed: 

\end{enumerate}

\subsubsection{Module ?}

...

\subsection{Traceability Between Test Cases and Modules}

\wss{Provide evidence that all of the modules have been considered.}
				
\bibliographystyle{plainnat}

\bibliography{../../refs/References}

\newpage

\section{Appendix}

This is where you can place additional information.

\subsection{Symbolic Parameters}

The definition of the test cases will call for SYMBOLIC\_CONSTANTS.
Their values are defined in this section for easy maintenance.

\subsection{Usability Survey Questions?}

\wss{This is a section that would be appropriate for some projects.}

\newpage{}
\section*{Appendix --- Reflection}

The information in this section will be used to evaluate the team members on the
graduate attribute of Lifelong Learning.  Please answer the following questions:

\newpage{}
\section*{Appendix --- Reflection}

The information in this section will be used to evaluate the team members on the
graduate attribute of Lifelong Learning.  Please answer the following questions:

\begin{enumerate}
  \item What knowledge and skills will the team collectively need to acquire to
  successfully complete the verification and validation of your project?
  Examples of possible knowledge and skills include dynamic testing knowledge,
  static testing knowledge, specific tool usage etc.  You should look to
  identify at least one item for each team member.
  \item For each of the knowledge areas and skills identified in the previous
  question, what are at least two approaches to acquiring the knowledge or
  mastering the skill?  Of the identified approaches, which will each team
  member pursue, and why did they make this choice?
\end{enumerate}

\end{document}
