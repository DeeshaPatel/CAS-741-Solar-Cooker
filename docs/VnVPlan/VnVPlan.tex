\documentclass[12pt, titlepage]{article}

\usepackage{booktabs}
\usepackage{tabularx}
\usepackage{colortbl}
\usepackage{hyperref}
\hypersetup{
    colorlinks,
    citecolor=blue,
    filecolor=black,
    linkcolor=red,
    urlcolor=blue
}
\usepackage[round]{natbib}



\begin{document}

\title{System Verification and Validation Plan for \progname{Solar Cooker Energy Calculator}} 
\author{Deesha Patel}
\date{\today}
	
\maketitle

\pagenumbering{roman}

\section{Revision History}

\begin{tabularx}{\textwidth}{p{5cm}p{1.5cm}X}
\toprule {\bf Date} & {\bf Version} & {\bf Notes}\\
\midrule
February 14, 2023 & 0.1.0 & Add General Information section \\
& 0.2 & Add further details in different section \\

February 18, 2023 & 1.0 & First Draft of VnV \\

\bottomrule
\end{tabularx}

\newpage

\tableofcontents

\listoftables


\newpage

\section{Symbols, Abbreviations and Acronyms}

\renewcommand{\arraystretch}{1.2}
\begin{tabular}{l l} 
  \toprule		
  \textbf{symbol} & \textbf{description}\\
  \midrule 
  MG & Module Guide \\
  MIS & Module Interface Specification \\
  SRS & Software Requirement Specification\\
  SCEC & Solar Cooker Energy Calculator \\ 
  TC & Test Case \\
  VnV & Verification and Validation \\ 
  
  \bottomrule
\end{tabular}\\

For complete symbols used within the system, please refer the section 1 in 
  \href{https://github.com/DeeshaPatel/CAS-741-Solar-Cooker/blob/7c53c8d9a19ca2f94dfba6ba9208eae0bf03b8cc/docs/SRS/SRS.pdf}{SRS} document.


\newpage

\pagenumbering{arabic}

This document provides the road-map of the Verification and Validation plan for Solar Cooker Energy Calculator for ensuring the requirements and goals of the program (found in \href{https://github.com/DeeshaPatel/CAS-741-Solar-Cooker/blob/7c53c8d9a19ca2f94dfba6ba9208eae0bf03b8cc/docs/SRS/SRS.pdf}{SRS} document). The organization of this document starts with the General Information about the Solar Cooker Energy Calculator in \autoref{GeneralInformation}. A verification plan is provided in \autoref{plan} and \autoref{systemtests} describes the system tests, including tests for functional and non-functional requirements.
  

\section{General Information}
\label{GeneralInformation}

\subsection{Summary}

This document reviews the validation and verification plan for Solar Cooker Energy Calculator (SCEC), a program that calculate the balance temperature at recipient and cooking power in it using user inputs.  

\subsection{Objectives}

The purpose of the validation plan is to define how system validation will perform at the end of the project - the strategy will use to assess whether the developed system accomplishes the designed goals. Also, the verification plan includes test strategies, definitions of what will be tested, and a test matrix with detailed mapping connecting the testing performed to the system requirements. This verification plan ensures that all requirements specified in the System Requirements Specification(\href{https://github.com/DeeshaPatel/CAS-741-Solar-Cooker/blob/7c53c8d9a19ca2f94dfba6ba9208eae0bf03b8cc/docs/SRS/SRS.pdf}{SRS}) document have been met and reviewed. The specific goal of this document is to demonstrate the adequate usability of the system.


\subsection{Relevant Documentation}

The relevant documentation for the SCEC includes \href{https://github.com/DeeshaPatel/CAS-741-Solar-Cooker/blob/7c53c8d9a19ca2f94dfba6ba9208eae0bf03b8cc/docs/ProblemStatementAndGoals/ProblemStatement.pdf}{Problem Statement}, \href{https://github.com/DeeshaPatel/CAS-741-Solar-Cooker/blob/7c53c8d9a19ca2f94dfba6ba9208eae0bf03b8cc/docs/SRS/SRS.pdf}{System Requirements Specifications}, VnV Report, MG and MIS (found in \href{https://github.com/DeeshaPatel/CAS-741-Solar-Cooker/tree/main/docs}{Github Repository})  


\section{Plan}
\label{plan}

This section describes the plan for the Solar Cooker Energy Calculator system. The planning starts with the Verification and Validation team, followed by the SRS verification plan, Design verification plan, Implementation verification plan, Automated testing and verification tools, and Software validation plan.  

\subsection{Verification and Validation Team}

This section describes the members of Verification and Validation plan. 

\begin{center}
\begin{table}[h!]
\begin{tabular}{ |l|l|p{2cm}|p{5cm}| } 
\hline
\rowcolor[gray]{0.9}
\textbf{Name} & \textbf{Document} & \textbf{Role} & \textbf{Description} \\
\hline
 Dr. Spencer Smith & all & Instructor/ Reviewer & Review the documents, design and documentation style. \\ 
 \hline
 Deesha Patel & all & Author & Create all the documents, provide the VnV plan, test case and test execution, verify the implementation. \\  
 \hline
 Mina Mahdipour & all & Domain Expert Reviewer & Review all the documents and review the VnV plan. \\  
 \hline
 Karen Wang & SRS & Secondary Reviewer & Review the SRS document \\
 \hline
Lesley  Wheat & VnV Plan & Secondary Reviewer & Review the VnV plan. \\ 
\hline 
Sam Joseph Crawford & MG + MIS & Secondary Reviewer & Review the MG and MIS document. \\
\hline 

\hline
\end{tabular}
\caption{Verification and Validation team} 
\end{table}
\end{center}

\subsection{SRS Verification Plan}

The SCEC SRS document shall be verified in the following way: 

\begin{enumerate}

\item Initial review from the assigned members (Dr. Spencer Smith, Mina Mahdipour, Karen Wang, and Deesha Patel) will be performed. For this, the manual review will perform using the given \href{https://github.com/smiths/capTemplate/blob/9251702fdcb9800c59f6ed3d11d91e2bd62fca6d/docs/Checklists/SRS-Checklist.pdf}{SRS Checklist}, designed by Dr. Smith. 

\item Reviewer can give feedback to the author by creating the issue in Github. 

\item Author (Deesha Patel) is responsible to address the issues created by the primary and secondary reviewers. Also, need to address the suggestions given by the instructor (Dr. Spencer Smith).  

\end{enumerate}


\subsection{Design Verification Plan}

The design documents, Module Guide (MG), and Module Interface Specification (MIS) will be verified through the static technic of document inspection by the Domain/ Primary expert (Mina Mahdipour) and Secondary Reviewer (Sam Joseph Crawford). Also, the class instructor (Dr. Spencer Smith) will review both documents. Reviewers can give feedback to the author by creating the issue in Github. The author is responsible to solve the issues and address the suggestions. The reviewer will assess this document with the help of \href{https://github.com/smiths/capTemplate/blob/9251702fdcb9800c59f6ed3d11d91e2bd62fca6d/docs/Checklists/MG-Checklist.pdf}{MG Checklist} and \href{https://github.com/smiths/capTemplate/blob/9251702fdcb9800c59f6ed3d11d91e2bd62fca6d/docs/Checklists/MIS-Checklist.pdf}{MIS Checklist} designed by Dr. Smith.      

\subsection{Implementation Verification Plan}

The implementation of SCEC shall be verified in the following ways:

\begin{itemize}

\item Static testing for SCEC:

\begin{itemize}


\item Code Walkthrough: This process will be performed by the author (Deesha Patel) and Domain expert (Mina Mahdipour). An author will share the copy of the original code with Domain expert and then domain expert will manually test the code with different test cases. The domain expert will raise the issue in GitHub if finds any issue with the code.

\end{itemize}
\end{itemize}

\begin{itemize}

\item Dynamic testing for SCEC:

\begin{itemize}


\item Test cases: Test cases for all the mentioned tests in \autoref{systemtests} will be carried out. These tests target functional and non-functional requirements listed in the \href{https://github.com/DeeshaPatel/CAS-741-Solar-Cooker/blob/7c53c8d9a19ca2f94dfba6ba9208eae0bf03b8cc/docs/SRS/SRS.pdf}{SRS} document. All the test cases are manual or automatic.  

\end{itemize}
\end{itemize}



\subsection{Automated Testing and Verification Tools}

\begin{itemize}
    \item System and Unit tests: Automated testing of SCEC is conducted using Pytest library in Python. These tests are performed by predetermining user inputs and comparing it with expected values. 
\end{itemize}

\subsection{Software Validation Plan}

Software validation plan is beyond the scope for SCEC System.

\section{System Test Description}
\label{systemtests}
	
\subsection{Tests for Functional Requirements}

Functional requirements for SCEC are given in \href{https://github.com/DeeshaPatel/CAS-741-Solar-Cooker/blob/7c53c8d9a19ca2f94dfba6ba9208eae0bf03b8cc/docs/SRS/SRS.pdf}{SRS} section 5.1. Some input values are taken from the paper~\cite{MathsModel}. There are 5 functional requirements for SCEC, R1 and R2 are related to the inputs, while R3 to R5 are corresponding to outputs. \autoref{input_functional_tests} describes the input tests related to R1 and R2; and \autoref{output_functional_tests} describes the output tests for R3 to R5.        

\subsubsection{Input tests}
\label{input_functional_tests}

\paragraph{Functional tests - Input tests - Area of object}


\begin{table}[h!]
\begin{center}
\begin{tabular}{ lccccc }
\hline
\multicolumn{1}{l|}{}   & \multicolumn{3}{c|}{Input}                            & \multicolumn{2}{c}{Output} \\ 

\hline

\multicolumn{1}{c|}{ID} &   ${A_t}$   &   $A_\text{ref}$   &   \multicolumn{1}{c|}{${A_{\text{m}}}$}   &   $valid?$   &   $Error Message$ \\ \hline

TC-SCEC-1-1   &   0.039  & 0.046  & 0.064    &  Y  & NONE                         \\
TC-SCEC-1-2   &   0      & 0.037  & 0.059    &  N  & Non-zero required            \\
TC-SCEC-1-3   &   0.67   & 0.0942 & 0        &  N  & Non-zero required            \\ 
TC-SCEC-1-4   &   0.741  & 0      & 0.0424   &  N  & Non-zero required            \\
TC-SCEC-1-5   &   -0.063 & 0.728  & 0.572    &  N  & Positive value required      \\
TC-SCEC-1-6   &   0.025  & -0.279 & 0.763    &  N  & Positive value required      \\
TC-SCEC-1-7   &   0.025  & 0.279  & -0.763   &  N  & Positive value required      \\
TC-SCEC-1-8   &   1000   & 0.245  & 0.562    &  N  & Too long area input          \\
TC-SCEC-1-9   &   0.562  & 0.285  & 0.13f    &  N  & Alphabet not allowed         \\ 

\hline


\end{tabular}
\caption{TC-SCEC-1 - Area input constraints tests}
\label{tab:tc-SCEC-1}
\end{center}
\end{table}


\begin{enumerate}

\item{test-id1: Valid Area inputs  \\}

Control: Automatic
					
Initial State: Pending input 
					
Input: Set of input values for area of particular object given in the \autoref{tab:tc-SCEC-1}. 
					
Output: Give an appropriate error message defined in the \autoref{tab:tc-SCEC-1}. 

Test Case Derivation: This test case is to test the behaviour of the system when the system is supplied with inputs for area that are the physical constraints of Solar cooker box. In test cases TC-SCEC-1-2 to TC-SCEC-1-9, the system produces the error message, as those are invalid inputs.     
					
How test will be performed: The automatic test is performed using PyTest.  

\end{enumerate}

\paragraph{Functional tests - Input tests - Temperature value}


\begin{table}[h!]
\begin{center}
\begin{tabular}{ lccccccc }
\hline
\multicolumn{1}{l|}{}   & \multicolumn{5}{c|}{Input}                            & \multicolumn{2}{c}{Output} \\ 

\hline

\multicolumn{1}{c|}{ID} &   ${T_t}$   &   ${T_\text{g_2}}$  &  $T_f$ & ${T_\text{init}}$ &   \multicolumn{1}{c|}{${T_{\text{ref}}}$}   &   {valid?}   &   Error Message    \\ \hline

TC-SCEC-2-1   &   30  & 30.2 & 32.5   & 40.1  &  41.3    &  Y  & NONE                       \\
TC-SCEC-2-2   &   0  & 24.5  &  41.3  & 24.1  & 51.3     &  N  & Non-zero required           \\
TC-SCEC-2-3   &   12.1 & 24.6 & 56.2  & 43.2  & -13.4    &  N  & Positive temperature required      \\
TC-SCEC-2-4   &  23.5 & 26.4 & 26.6 & 36.2 & & N & Empty temperature value \\ 
TC-SCEC-2-5   &  24.5 & 26.8 & 210.3 & 25.7 & 29.4 & N & Exceed temperature value \\ 


\hline


\end{tabular}
\caption{TC-SCEC-2 - Temperature input constraints tests}
\label{tab:tc-SCEC-2}
\end{center}
\end{table}

\begin{enumerate}   

\item{test-id1: Valid/Invalid Temperature value  \\}

Control: Automatic
					
Initial State: Pending input 
					
Input: Pass the value of temperature specified input column in the \autoref{tab:tc-SCEC-2}.
					
Output: verify the output of the software matches the output column specified in \autoref{tab:tc-SCEC-2}. 

Test Case Derivation: This test case is to test the behaviour of the system when the system is supplied with inputs for temperature. In test cases TC-SCEC-1-2 to TC-SCEC-1-5, the system produces the error message, as those are invalid inputs. 
					
How test will be performed: The automatic test is performed using PyTest.  


					

\end{enumerate}

\paragraph{Functional tests - Input tests - other parameters}


\begin{table}[h!]
\begin{center}
\begin{tabular}{ lcccc }
\hline
\multicolumn{1}{l|}{}   & \multicolumn{2}{c|}{Input}                            & \multicolumn{2}{c}{Output} \\ 

\hline

\multicolumn{1}{c|}{ID}  &   $\epsilon_\text{ref}$  &   \multicolumn{1}{c|}{$\epsilon_t$}   &   {valid?}   &   Error Message    \\ \hline

TC-SCEC-3-1   &   1      & 0.95   & Y   & NONE          \\
TC-SCEC-3-2   &   0.97   & 0.91   & Y   & NONE          \\
TC-SCEC-3-3   &   0.93   & 1.3    & N   & Not in range  \\
TC-SCEC-3-4   &   -0.75  & 0.86   & N   & Not in range  \\ 

\hline


\end{tabular}
\caption{TC-SCEC-3 - Other input constraints tests}
\label{tab:tc-SCEC-3}
\end{center}
\end{table}

\begin{enumerate}   



\item{test-id1: Valid/Invalid Emittance value of object  \\}

Control: Automatic
					
Initial State: Pending input 
					
Input: Pass the value of temperature specified input column in the \autoref{tab:tc-SCEC-3}.
					
Output: verify the output of the software matches the output column specified in \autoref{tab:tc-SCEC-3}. 

Test Case Derivation: This test case is to test the behaviour of the system when the system is supplied with inputs for temperature. In test cases TC-SCEC-1-2 to TC-SCEC-1-5, the system produces the error message, as those are invalid inputs. 
					
How test will be performed: The automatic test is performed using PyTest.  

\end{enumerate}


\subsubsection{Output tests}
\label{output_functional_tests}
\begin{enumerate}

\item{test-id1: Validate the output of the fluid temperature in recipient  \\}

Control: Combination of manual and automatic 
					
Initial State: N/A
					
Input: Pass the input values: \\ 
       $A_m$ = 1.5 \\
       $A_t$ = 0.0201 \\
       $A_{\text{ref}}$ = 0.0058 \\
       $T_g2$ = 30 \\
       $T_t$ = 30 \\
       $T_f$ = 30 \\
       $T_\text{ref}$ = 30 \\ 
       $\epsilon_r$ = 1 \\
       $\epsilon_t$ = 0.85 \\
       
					
Output: Below output should be generated for each of the valid and real inputs. 
\begin{itemize}
    \item $T_r$ $\approx$ 95.9. 
    \item For other inputs, it should calculate the temperature value of the fluid 
    \item Graph should be generated with the temperature of the fluid and recipient.  
\end{itemize} 

Test Case Derivation: This test case is to test the output of the system when the system is supplied with all valid inputs. This test case is derived from 3rd and 4th requirement in SRS document.  
					
How test will be performed: The automatic test is performed using PyTest.  


\item{test-id2: Validate the output of the energy temperature in fluid  \\}

Control: Combination of manual and automatic 
					
Initial State: N/A
					
Input: Pass the input value: $T_\text{init}$ = 30 
					
Output: As an output, algorithm should calculate the non-negative and non-zero temperature energy value of the fluid.  

Test Case Derivation: This test case is to test the output of the system when the system is supplied with initial temperature of the fluid. This test case is derived from 5th requirement in SRS document.  
					
How test will be performed: The automatic test is performed using PyTest.  

\end{enumerate}   


\subsection{Tests for Nonfunctional Requirements}

Functional requirements for SCEC are given in \href{https://github.com/DeeshaPatel/CAS-741-Solar-Cooker/blob/7c53c8d9a19ca2f94dfba6ba9208eae0bf03b8cc/docs/SRS/SRS.pdf}{SRS} section 5.2. There are 5 functional requirements for SCEC. 


%\wss{Tests related to usability could include conducting a usability test and survey.  The survey will be in the Appendix.}

%\wss{Static tests, review, inspections, and walkthroughs, will not follow the format for the tests given below.}

\subsubsection{Non-functional: Understandability}
\label{non_functional_understandability}		
\paragraph{Understandability}

\begin{enumerate}

\item{test-id1: Understandability test\\}

Type: Manual
					
Initial State: None
					
Input/Condition: None
					
Output/Result: None 
					
How test will be performed: Domain Expert (Mina Mahdipour) will review the shared code and complete the survey mentioned in the \autoref{tab:tc-SCEC-4}.   

\begin{table}[h!]
\begin{center}
\begin{tabular}{ p{0.5cm}|p{10cm}|c }
\hline
No. &  Question   & Score(0-10) \\
\hline
1. & The whole code indented properly to understand the flow. & \\
2. & The name of the variables and method is meaningful. & \\
3. & Different comments is useful to understand the importance of code. &  \\
4. & Task are well broken into function. & \\
5. & Overall quality. & \\

\hline

\end{tabular}
\caption{TC-SCEC-4 - Understandability test survey}
\label{tab:tc-SCEC-4}
\end{center}
\end{table}

\end{enumerate}

\subsubsection{Non-functional: Maintainability}
\label{non_functional_maintainability}	
\paragraph{Maintainability}

\begin{enumerate}
					
\item{test-id1: Maintainability\\}

Type: Code walkthrough
					
Initial State: None 
					
Input: None 
					
Output: None
					
How test will be performed: During code walkthrough meeting all the details related to the software lifespan, coupling of the software architecture, documentation of the software will discussed among team members.

\end{enumerate}


\subsubsection{Non-functional: Usability}
\label{non_functional_usability}	
\paragraph{Usability}

\begin{enumerate}
					
\item{test-id1: Usability\\}

Type: Manual with group of people
					
Initial State: None 
					
Input: None 
					
Output: None
					
How test will be performed: The user group will be asked to install the software on their system and give input on their own. Then user need to fill out the short answer survey given in the \autoref{tab:tc-SCEC-5}

\end{enumerate}

\begin{table}[h!]
\begin{center}
\begin{tabular}{ p{0.5cm}|p{10cm}|c }
\hline
No. &  Question   & Answer ) \\
\hline
1. & Which operating system are you using?  & \\
2. & Is system running smoothly on your computer?  & \\
3. & Is invalid input's message clear?  &  \\
4. & Is software easy to use? & \\
5. & Is text easy to read? & \\
6. & What, if anything, surprised you about the experience? & \\
7. & What did you like the least? & \\
8. & Do you have any suggestion?  & \\

\hline
\end{tabular}
\caption{TC-SCEC-5 - Usability test survey}
\label{tab:tc-SCEC-5}
\end{center}
\end{table}

\subsubsection{Non-functional: Portability}
\label{non_functional_portability}		
\paragraph{Portability}

\begin{enumerate}
					
\item{test-id1: Portability\\}

Type: Manual
					
Initial State: None 
					
Input: None 
					
Output: None
					
How test will be performed: Code developer (Deesha Patel) will try to install and run whole software in different operating system. Also, need to ensure that all the given test cases pass in all different operating system.  

\end{enumerate}



\subsection{Traceability Between Test Cases and Requirements}

A traceability between test cases and requirements is shown in \autoref{tab:tc-traceability} 



\begin{table}[h!]
\begin{center}
\begin{tabular}{ l|c|c|c|c|c|c|c|c|c }
\hline
 & R1   & R2 & R3 & R4 & R5 & NFR1 & NFR2 & NFR3 & NFR4 \\
\hline
\ref{input_functional_tests} & X & X & & & & & & \\
\hline
\ref{output_functional_tests} &  &  & X & X & X & & & \\
\hline
\ref{non_functional_understandability} & & & & & & X & & & \\
\hline
\ref{non_functional_maintainability} & & & & & & & X & & \\
\hline
\ref{non_functional_usability} & & & & & & & & X & \\
\hline
\ref{non_functional_portability} & & & & & & & & & X \\

\hline
\end{tabular}
\caption{Tracebility between Test cases and Requirements}
\label{tab:tc-traceability}
\end{center}
\end{table}

\section{Unit Test Description}
This section is intensionally blank until MIS complete.

\newpage
\bibliographystyle {plain}
\bibliography {srs}

\end{document}
